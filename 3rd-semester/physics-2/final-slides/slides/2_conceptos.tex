
\section{CONCEPTOS}
\subsection{}

\begin{frame}{CONCEPTOS}
  \begin{itemize}
    \item \textbf{Onda electromagnética}: Son ondas que no necesitan un medio para propagarse. Están compuestas de un campo magnético y uno eléctrico, ortogonales entre sí.
    \item \textbf{Longitud de onda (\boldmath$\lambda$)}: Distancia recorrida en un periodo.
    \item \textbf{Frente de onda:} Planos en los que el campo magnético y eléctrico son constantes.
    \item \textbf{Luz visible}: Intervalo del espectro electromagnético de $\lambda$ entre 400nm y 750 nm. Para el caso de reflexión y refracción se considera como un haz, pero también funciona para el modelo ondulatorio.
    \item \textbf{Ángulo de incidencia}: Ángulo con el que llega la onda al objeto de estudio. Se mide desde la normal a éste.
    \item \textbf{Imagen virtual:} Es la imagen formada cuando los rayos salientes divergen.
  \end{itemize}
\end{frame}
%%%%%%%%%%%%%%%%%%%%%%%%%%%%%%%%%%%%%%%%%%%%%%%%%%%%%%%%%%
