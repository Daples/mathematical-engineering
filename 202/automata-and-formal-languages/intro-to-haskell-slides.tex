\documentclass{beamer}
\usepackage[english]{babel}
\usepackage{geometry}
\usepackage{amsmath}
\usepackage{amsthm}
\usepackage{graphicx}
\usepackage{caption}
\usepackage[utf8]{inputenc}

%%%%%%%% HYPERREF PACKAGE
\hypersetup{colorlinks=true}
\hypersetup{citecolor=orange}
\hypersetup{urlcolor=orange}

%%%%%%%% DEFINITION AND THEOREM DEFINITIONS
\let\definition\relax
\theoremstyle{definition}
\newtheorem{definition}{Definition}[section]

\let\remark\relax
\theoremstyle{remark}
\newtheorem{remark}{Remark}

\theoremstyle{example}
\newtheorem{question}{Question}

%%%%%%%% MULTI-COLUMNS PACKAGE
\usepackage{multicol}

%%%%%%%% BIB-LATEX STUFF
\usepackage[style=authoryear,
            bibstyle=authoryear,
            citestyle=authoryear,
            hyperref=true,
            backend=biber]{biblatex}
\addbibresource{~/juanscr/jsc/jsc/backups/ref.bib} % Bibliography file

%%%%%%%% ICONS FOR CITES
% https://tex.stackexchange.com/questions/68080/beamer-bibliography-icon
\setbeamertemplate{bibliography item}{%
  \ifboolexpr{ test {\ifentrytype{book}} or test {\ifentrytype{mvbook}}
    or test {\ifentrytype{collection}} or test {\ifentrytype{mvcollection}}
    or test {\ifentrytype{reference}} or test {\ifentrytype{mvreference}} }
    {\setbeamertemplate{bibliography item}[book]}
    {\ifentrytype{online}
       {\setbeamertemplate{bibliography item}[online]}
       {\setbeamertemplate{bibliography item}[article]}}
  \usebeamertemplate{bibliography item}}

\defbibenvironment{bibliography}
  {\list{}
     {\settowidth{\labelwidth}{\usebeamertemplate{bibliography item}}
      \setlength{\leftmargin}{\labelwidth}
      \setlength{\labelsep}{\biblabelsep}
      \addtolength{\leftmargin}{\labelsep}
      \setlength{\itemsep}{\bibitemsep}
      \setlength{\parsep}{\bibparsep}}}
  {\endlist}
  {\item}

%%%%%%%% BRACKETS AROUND CITE
% https://tex.stackexchange.com/questions/16765/biblatex-author-year-square-brackets
\makeatletter

\newrobustcmd*{\parentexttrack}[1]{
  \begingroup
  \blx@blxinit
  \blx@setsfcodes
  \blx@bibopenparen#1\blx@bibcloseparen
  \endgroup}

\AtEveryCite{
  \let\parentext=\parentexttrack
  \let\bibopenparen=\bibopenbracket
  \let\bibcloseparen=\bibclosebracket}

\makeatother

%%%%%%%% THEME SLIDES
\usetheme{default}

%%%%%%%% BEAMER STUFF
\setbeamertemplate{footline}[frame number]
\setbeamertemplate{section in toc}[sections numbered]
\setbeamertemplate{subsection in toc}[subsections numbered]
\setbeamertemplate{navigation symbols}{}

% Not enumerate frame breaks
% https://tex.stackexchange.com/questions/295854/how-to-edit-behaviour-of-frame-titles-during-frame-break-in-beamer
\setbeamertemplate{frametitle continuation}[from second][]

%%%%%%%% FRAME TITLES AND SUBTITLES
% https://stackoverflow.com/questions/35810906/vim-beamer-dynamic-frame-title1
\newif\ifinsection
\newif\ifinsubsection

\let\oldsection\section
\renewcommand{\section}{
  \global\insectiontrue
  \global\insubsectionfalse
  \oldsection}
\let\oldsubsection\subsection
\renewcommand{\subsection}{
  \global\insubsectiontrue
  \oldsubsection}

\newcommand {\aframe}[1] {
  \begin{frame}
    \ifinsection\frametitle{\secname}\fi
    \ifinsubsection\framesubtitle{\subsecname}\fi
  #1
  \end{frame}
}

% Blue line after title
% https://tex.stackexchange.com/questions/343517/beamer-full-width-hrule-below-frame-title
\setbeamertemplate{frametitle}{%
    \usebeamerfont{frametitle}\insertframetitle\strut%
    \vskip.0\baselineskip%
    \leaders\vrule width 0.85\paperwidth\vskip0.4pt%
    \vskip2pt%
    \usebeamerfont{framesubtitle}\insertframesubtitle
}

%%%%%%%% FOOTNOTE STUFF
\renewcommand{\thefootnote}{\fnsymbol{footnote}}
% Footnote without symbol
% https://tex.stackexchange.com/questions/30720/footnote-without-a-marker
\newcommand\blfootnote[1]{%
  \begingroup
  \renewcommand\thefootnote{}\footnote{#1}%
  \addtocounter{footnote}{-1}%
  \endgroup
}

%%%%%%%% CENTER OF SLIDE THANK YOU
\usepackage{tikz}

%%%%%%%% SUB-FIGURE PACKAGE
\usepackage{subcaption}

%%%%%%%% PERSONAL COMMANDS
\usepackage{amssymb}

%%%% Important sets
\renewcommand{\O}{\mathbb{O}}
\newcommand{\N}{\mathbb{N}}
\newcommand{\Z}{{\mathbb{Z}}}
\newcommand{\Q}{{\mathbb{Q}}}
\newcommand{\R}{{\mathbb{R}}}

%%%% Usual operations
\newcommand{\pow}[2]{#1^{#2}}
\newcommand{\expp}[1]{e^{#1}}
\newcommand{\fst}{\mathrm{fst}}
\newcommand{\snd}{\mathrm{snd}}

%%%% Lambda Calculus Symbols
\newcommand{\dneq}{\,\, \# \,\,}
\renewcommand{\S}{\pmb{\mathrm{S}}}
\newcommand{\I}{\pmb{\mathrm{I}}}
\newcommand{\K}{\pmb{\mathrm{K}}}
\newcommand{\ch}[1]{\ulcorner #1 \urcorner}

%%%% Ordinal Lambda Calculus Symbols
\newcommand{\ordAlph}{\Sigma_{\text{Ord}}}
\newcommand{\termOrd}{\text{Term}_\text{Ord}}
\newcommand{\fl}{\mathrm{fl}}
\newcommand{\sk}{\mathrm{sk}}

%%%% Superscript to the left
% https://latex.org/forum/viewtopic.php?t=455
\usepackage{tensor}
\newcommand{\app}[3]{\tensor*[^{#1}]{\left(#2, #3\right)}{}}

%%%% Make optional parameter
% https://tex.stackexchange.com/questions/217757/special-behavior-if-optional-argument-is-not-passed
\usepackage{xparse}

%%%% Statistics
\NewDocumentCommand{\E}{o m}{
  \IfNoValueTF{#1}
  {\mathbb{E}\left[#2\right]}
  {\mathbb{E}^{#1}\left[ #2\right]}
}
\NewDocumentCommand{\V}{o m}{
  \IfNoValueTF{#1}
  {\mathrm{Var}\left[#2\right]}
  {\mathrm{Var}^{#1}\left[ #2\right]}
}
\RenewDocumentCommand{\P}{o o m}{
  \IfNoValueTF{#1}
  {\IfNoValueTF{#2}
    {\mathrm{P}\left(#3\right)}
    {\mathrm{P}^{#2}\left(#3\right)}}
  {\IfNoValueTF{#2}
    {\mathrm{P}_{#1}\left(#3\right)}
    {\mathrm{P}_{#1}^{#2} \left(#3\right)}}
}

%%%% Lambda Calculus
\NewDocumentCommand{\cx}{o}{
  \IfNoValueTF{#1}
  {\left[\quad\right]}
  {\left[\, #1 \,\right]}
}

%%%% Create absolute value function
% https://tex.stackexchange.com/questions/43008/absolute-value-symbols
\usepackage{mathtools}
\DeclarePairedDelimiter\abs{\lvert}{\rvert}%
\DeclarePairedDelimiter\norm{\lVert}{\rVert}%
\makeatletter
\let\oldabs\abs
\def\abs{\@ifstar{\oldabs}{\oldabs*}}
%
\let\oldnorm\norm
\def\norm{\@ifstar{\oldnorm}{\oldnorm*}}
\makeatother

%%%%%%%% LOGIC TREES
\usepackage{prftree}

%%%%%%%% SPLIT EQUATIONS
% https://tex.stackexchange.com/questions/51682/is-it-possible-to-pagebreak-aligned-equations
\allowdisplaybreaks

% Better spacing between display equation
% https://bit.ly/2zLJEXN
\makeatletter
\g@addto@macro\normalsize{%
    \setlength\belowdisplayskip{8pt}
  }

%%%%%%%% DIFFERENT FONTS FOR MATH
\usepackage{mathrsfs}

%%%%%%%% TO USE SHORT COMMANDS FOR VECTOR LINES
\usepackage{esvect}

\usepackage{minted}
\usemintedstyle{borland}

%%%%%%%% START DOCUMENT
\title{Haskell Introduction}

\author{Juan Sebasti\'an C\'ardenas-Rodríguez \\
  \scalebox{0.7}{Mathematical Engineering, Universidad EAFIT}}

\begin{document}

% Plain so is not numbered
\begin{frame}[plain]
  \titlepage
\end{frame}

%%%%%%%%%% SLIDES %%%%%%%%%%%%%%%
\section{What is Haskell?}
\aframe{%
  \begin{figure}[H]
    \centering
    \includegraphics[scale=.4]{figs/haskell-logo}
  \end{figure}%
  {\small\textit{...you don't tell the computer what to do as such but rather
      you tell it what stuff is.} \parencite{lipovaca2011}} \pause
  %
  \begin{multicols}{2}
    \noindent
    \textbf{Advantages}
    \begin{itemize}
      \item Statically typed.
      \item Lazy evaluated.
      \item Efficient garbage collection.
      \item Elegant and concise code.
    \end{itemize} \pause
    %
    \columnbreak
    %
    \textbf{Difficulties}
    \begin{itemize}
      \item \texttt{for}, \texttt{while}, \texttt{do-while} etc... do not
      exist.
      \item It's \textit{different} than commonly used languages such as Python,
        Java, etc...
    \end{itemize}
  \end{multicols}
}

\begin{frame}[fragile]
  \frametitle{Haskell Code}
  \framesubtitle{Factorial Function}
  \begin{example}
    \vspace{-0.4cm}
    \begin{multicols}{2}
      \noindent \textit{Function}%
      \begin{flalign*}
        n! =
        \begin{cases}
          1,&n = 0 \\
          n * (n - 1)!, &\text{otherwise}
        \end{cases}&&
      \end{flalign*}
      \vfill\pause
      \textit{Haskell}%
      \inputminted{haskell}{codes/factorial.hs}
    \end{multicols}
  \end{example}
  \pause
  \begin{question}
    Is the previous function correct?
  \end{question} \pause
  %
  \inputminted{haskell}{codes/factorial-no-error.hs}
\end{frame}

\begin{frame}[fragile]
  \frametitle{Haskell Code}
  \framesubtitle{Syntax}
  \textbf{Guards}%
  \vspace{-0.3cm}
  {\setlength{\columnsep}{-40pt}
    \begin{multicols}{2}
      \inputminted{haskell}{codes/sgn-guards.hs} \vfill
      \inputminted{haskell}{codes/factorial-no-error.hs}
    \end{multicols}}
  %
  \vspace{0.3cm}\pause
  \textbf{Pattern matching}
  \inputminted{haskell}{codes/pattern-matching.hs}
\end{frame}

\begin{frame}[fragile]
  \frametitle{Haskell Code}
  \framesubtitle{Syntax}
  \textbf{Let clauses}
  \inputminted{haskell}{codes/fibo-let.hs} \pause
  %
  \textbf{Where clauses}%
  \inputminted{haskell}{codes/fibo-where.hs}
\end{frame}

\section{Higher Order Functions}
\subsection{Currying}
\aframe{
  \begin{quote}
    Currying is the process of transforming a function that takes multiple
    arguments in a tuple as its argument, into a function that takes just a
    single argument and returns another function which accepts further
    arguments...\footnote{\href{https://wiki.haskell.org/Currying}{Haskell
        wiki}}
  \end{quote} \pause
  \begin{multicols}{2}
    \noindent
    Original function
    \begin{flalign*}
      f: \R \times \R &\rightarrow \R \\
      (x, y) &\mapsto f(x,y) = x + y&&
    \end{flalign*} \pause
    \vfill\columnbreak \noindent Curried form
    \vspace{-0.2cm}
    \only<3>{\begin{flalign*}
      f: \R &\rightarrow (\R \rightarrow \R) \\
      x &\mapsto f(x) = y \mapsto x + y&&
    \end{flalign*}}
    \only<4->{\begin{flalign*}
      f: \R &\rightarrow \R \rightarrow \R \\
      x &\mapsto f(x) = y \mapsto x + y&&
    \end{flalign*}}
\end{multicols}
\vspace{-0.8cm}
\onslide<5->{
\begin{question}
  What are the advantages of currying a function?
\end{question}}
}

\begin{frame}[fragile]
  \frametitle{Higher Order Functions}
  \framesubtitle{Currying}
  Multi-argument functions in Haskell.
  \inputminted{haskell}{codes/max.hs} \pause
  %
  \vspace{0.2cm}
  Partial application
  \inputminted{haskell}{codes/partial-max.hs} \pause
  %
  \vspace{0.2cm}
  Returning a function
  \inputminted{haskell}{codes/func-sum.hs}
\end{frame}

\begin{frame}[fragile]
  \frametitle{Built-in Data Types}
  \framesubtitle{Lists}
  A list in Haskell is defined inductively:
  \begin{itemize}
    \item The empty list \texttt{[]} is a list.
    \item The list by the first element \texttt{x} and another list
      \texttt{xs}, denoted by \texttt{(x:xs)}, is a list.
    \end{itemize}
    i.e. \texttt{[0,1]} is represented in Haskell as \texttt{(0:(1:[]))}. \pause
  \begin{example}
    How to calculate the length of a list?
    \begin{overlayarea}{\textwidth}{0.17\textheight}
      \only<2>{\inputminted{haskell}{codes/length-inc.hs}}
      \only<3>{\inputminted{haskell}{codes/length.hs}}
      \only<4->{\inputminted{haskell}{codes/length-poly.hs}}
    \end{overlayarea}
  \end{example}
  %
  \onslide<5->{See more basic functions of lists in
    \href{https://bit.ly/3ftdlvl}{Data.List}.}
\end{frame}

\begin{frame}[fragile]
  \frametitle{Algebraic Data Types}
  \framesubtitle{Bool}%
  Let's define the boolean type and some basic functions (Haskell has a built-in
  data type). \pause
  %
  \inputminted{haskell}{codes/bool-test.hs}
\end{frame}

\begin{frame}[fragile]
  \frametitle{Lazy Evaluation}
  \begin{question}
    What is the result of \texttt{foo 1}?
    \inputminted{haskell}{codes/foo.hs}
  \end{question} \pause
  %
  \begin{question}
    What is the result of \texttt{bar 1}?
    \begin{multicols}{2}
      \onslide<2->{\noindent \inputminted{haskell}{codes/bar.hs}}
      \columnbreak
      \onslide<4->{\inputminted{haskell}{codes/bar-non-term.hs}}
    \end{multicols}
    \onslide<3->{Thanks to lazy evaluation, it happens that \texttt{bar 1 =
        True}!}
    \onslide<5->{\\The function \texttt{bar0} does not terminate.}
  \end{question}
\end{frame}

\section{Recommendations}
\aframe{Books and community help:
  \begin{itemize}
    \item \href{https://wiki.haskell.org/Haskell}{Haskell wiki}
    \item \href{https://hackage.haskell.org/}{Hackage}
    \item Reddit, StackOverflow, etc...
    \item Some books recommendations\textcite{lipovaca2011},%
      \textcite{hutton2016} and\textcite{osullivan2008}.
  \end{itemize}
  \vspace{1cm}
  Installation recommendations:
  \begin{itemize}
    \item \textbf{Windows}: Use chocolatey to install latest version.
    \item \textbf{Unix-based systems}: Install GHCup.
  \end{itemize}
}

%%%%%%%%%%%%%%%%%%%%%%%%%%%%%%%%%
\begin{frame}[allowframebreaks]{References}
  \printbibliography
\end{frame}

\begin{frame}
  \begin{tikzpicture}[overlay, remember picture]
     \node[anchor=center] at (current page.center) {
     \Huge{\emph{Thank you!}}};
\end{tikzpicture}
  \begin{minipage}[t][.8\textheight]{\textwidth}
    \vfill
    \begin{center}
        \scalebox{0.7}{Juan Sebasti\'an C\'ardenas-Rodríguez} \\
        \scalebox{0.7}{jscardenar@eafit.edu.co} \\
    \end{center}
  \end{minipage}
\end{frame}

\end{document}
