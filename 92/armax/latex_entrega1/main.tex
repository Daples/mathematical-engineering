\documentclass[conference]{IEEEtran}
\IEEEoverridecommandlockouts
% The preceding line is only needed to identify funding in the first footnote. If that is unneeded, please comment it out.
\usepackage[utf8]{inputenc}
\usepackage[english]{babel}
\usepackage{cite}
\usepackage{amsmath,amssymb,amsfonts}
\usepackage{physics}
\usepackage{dsfont}
\usepackage{algorithmic}
\usepackage{graphicx}
\usepackage{subcaption}
\usepackage{array}
\usepackage{float}
\usepackage{lipsum}
\usepackage{textcomp}
\usepackage{xcolor}
\usepackage{diagbox}

\usepackage[hidelinks]{hyperref}
\def\BibTeX{{\rm B\kern-.05em{\sc i\kern-.025em b}\kern-.08em
    T\kern-.1667em\lower.7ex\hbox{E}\kern-.125emX}}
\begin{document}

\title{Simulation of ARMAX model for Forecast of Power Output of PV Grid\\}

\author{\IEEEauthorblockN{Juan S. C\'ardenas R.}
\IEEEauthorblockA{\textit{Student} \\
\textit{Universidad EAFIT}\\
Medell\'in, Colombia\\
jscardenar@eafit.edu.co}
\and
\IEEEauthorblockN{David Plazas E.}
\IEEEauthorblockA{\textit{Student} \\
\textit{Universidad EAFIT}\\
Medell\'in, Colombia \\
dplazas@eafit.edu.co}
}

\maketitle

\begin{abstract}
This work is devoted to the simulation of an ARMAX model proposed in the literature for better forecasting of power output of a Photo-Voltaic (PV) grid; the model includes information of environmental inputs (average temperature, precipitation amount, insolation duration, humidit) that classical time series approaches did not include. The simulation is performed using three different noise distributions in order to establish a comparison of the time series.
\end{abstract}

\begin{IEEEkeywords}

\end{IEEEkeywords}
\section{Introduction}\label{sec:intro}
Autoregressive Moving-Average (ARMA) models are widely used in areas such as economics and finance, with applications in modelling time series. The main problems with ARMA models is the estimation of its parameters, using only state data. One important remark is that ARMA models do not consider an observer, which cannot be always assumed in real conditions.

Classical parameter estimation of ARMA models is often addressed with standard statistical methods, for instance, instrumental least-square (ILS) method. The main problem with the ILS method is the requirement of the complete state of the real system, which is not always achievable since the system is perceived only through the observer.


This work is focused on the design of a Kalman-based filter for the estimation of ARMA parameters through the output of the observer, taking into consideration that the disturbances in the ARMA models can be correlated and this violates one of the main hypothesis of the standard Kalman-Filter (KF). We propose in this work a filter based on instrumental variables and extended KF.

\section{Problem Formulation}\label{sec:probForm}
As previously mentioned, the output of the PV system is an stochastic process. The standard approach to forecasting the behavior of this system has been modelled using ARIMA models, using only information of the past of the same system, but it does not take into consideration the external environmental factors that may affect the power output \cite{li2014armax}; as for the problem of this particular work, the main objective is to simulate the ARMAX model proposed in \cite{li2014armax} with different noise distributions, particularly with outlier behavior and show that it is not always adequate to assume normality of some random processes.

\section{Theoretical Approach}\label{sec:theo}
As previously mentioned, this works is devoted to the simulation and short analysis of an ARMAX system for ; hereby, the general ARMAX model is presented:
\begin{equation}
    z_{t+1}=\sum_{i=0}^{h_1}a_iz_{t-i}+\sum_{i=0}^{h_2}b_iu_{t-i}+\sum_{i=0}^{h_3}c_i\xi_{t-i}
\end{equation}
where $u_k$ are external inputs and $\xi_k$ are random noises. In this specific case, the model obtained in \cite{li2014armax}, 
\begin{equation}
\begin{aligned} Y_{t}=& 237.565+0.426 Y_{t-1}+a_{t}-0.153 a_{t-1}+8.9087 d_{1, t} \\ &-1.557 d_{7, t}+31.919 d_{8, t}-2.045 d_{9, t} \end{aligned}
\end{equation}
\section{Numerical Aspects}\label{sec:numAsp}
\subsection{ARMAX Model}
In this particular case, the model obtained in \cite{li2014armax} is:
\begin{equation}
\begin{aligned} z_{t}=& 237.565+0.426z_{t-1}+\xi_{t}-0.153 \xi_{t-1}+8.9087u_{1, t} \\ &-1.557 u_{7, t}+31.919 u_{8, t}-2.045u_{9, t}
\end{aligned}
\end{equation}
where $z_t$ is the power output of the PV grid in Watts (W); $u_{1,t}$ is the daily average temperature, $u_{7,t}$ is the precipitation amount, $u_{8,t}$ is the insolation duration and $u_{9,t}$ is the humidity.

\section{Numerical Results}\label{sec:numRes}
In the following figures, the simulations of the model for different distributions can be found.
\begin{figure}[h]
  \centering
  \includegraphics[scale=0.3]{files/sims.pdf}
  \caption{Results using Normal and Cauchy distribution.}
\end{figure}

\begin{figure}[h]
  \centering
  \includegraphics[scale=0.3]{files/simt.pdf}
  \caption{Results using Normal and T-Student distribution.}
\end{figure}



\section{Conclusions}\label{sec:conc}
In this article, an ARMAX model was successfully
simulated. Furthermore, it was seen the impact that different
distributions for the stochastic noise can have in the behavior of a
time series. At the same time, it is important to remark the
importance of historical data to make comparisons as the model in this
paper did not have similar values to the one in \cite{li2014armax}. This probably occurred because the values for the noise that the original paper used are not specified.


\nocite{*}
\bibliography{bib}
\bibliographystyle{IEEEtran}
\end{document}
