\section{Preliminaries}\label{sec:pre}
\subsection{ARMA Model}
The general scheme for the observer-ARMA model is given by
\begin{equation}
  \begin{split}
    &z_{t+1}=\sum_{s=0}^{l_1}a_sz_{t-s}+\sum_{s=0}^{l_2}b_s\xi_{t-s}\\
    &y_t=h_tz_t+\zeta_t\\
    &z_0\in\R, \ t=0,1,\dots
  \end{split}
\end{equation}
where $\xi_t$ and $\zeta_t$ are random perturbations.

\subsection{The Kalman-Filter State-Observer Model}
The KF state-observer model, given by
\begin{equation}\label{eq:kf_so}
  \begin{split}
    &x_{t+1}=F_tx_t + \xi_t\\
    &y_t=H_tx_t+\zeta_t\\
    &x_0\in\R^n, \ t=0,1,\dots
  \end{split}
\end{equation}
where $\xi_t$ and $\zeta_t$ are random noises. Every ARMA model can be translated into a form of system (\ref{eq:kf_so}).



\subsection{Traditional KF}
We will present, first, some notation in order to write the instrumental KF:
\begin{itemize}
  \item Observed data vector: $y^*_t:=(y_0^T,\,y_1^T,\dots,y_t^T)^T$.
  \item $z_{t|t-1}:=E(z_t|y^*_{t-1})$ and $y_{t|t-1}:=E(y_t|y^*_{t-1})$.
  \item $\Delta y_t:=y_t-y_{t|t-1}$ and $\Delta z_t:=z_t-z_{t|t-1}$.
  \item $\Sigma_{t|t}:={\rm Cov}(\Delta z_t|y_t^*$).
\end{itemize}
The standard KF
Procedure \cite{kalman1960new} has two main hypothesis:\\
\begin{itemize}
  \item \textbf{H1: }\begin{itemize}
    \item The noise perturbations must be white noise: \[E(\xi_t)=E(\zeta_t)=0, \quad t=0,1,\ldots\].
    \item Given finite second moments: \[E(\xi_t\xi_t^T)=Q_t, \quad E(\zeta_t\zeta_t^T)=R_t\].
    \item The perturbations are uncorrelated: \[E(\xi_t\xi_s^T)=E(\zeta_t\zeta_s^T)=0,\quad\forall s\neq t\]\[E(\xi_t\zeta_s^T)=0, \quad \forall t,s=0,1,\ldots\].
    \item The states are uncorrelated:
    \[E(z_tz_s^T)=0,\quad\forall t\neq s\]
  \end{itemize}
  \item \textbf{H2: }
  \begin{itemize}
    \item Normality of the perturbations:
    \[\xi_t\sim\mathcal{N}(0,Q_t),\quad\zeta_t\sim\mathcal{N}(0,R_t),\quad\forall t=0,1,\ldots\]
  \end{itemize}
\end{itemize}
The KF procedure is now presented:
\begin{enumerate}
  \item \textbf{Initialization: }
  \[
  z_{0|0}=z_0\in\R^n, \ \Sigma_{0|0}=0
  \]
  \item \textbf{Prediction: }
  \[
  \begin{split}
    &z_{t|t-1}=F_tz_{t-1|t-1}\\
    &\Sigma_{t|t-1}=F_{t-1}\Sigma_{t-1|t-1}F_{t-1}^T+Q_{t-1}
  \end{split}
  \]
  \item \textbf{Correction: }
  \[
  \begin{split}
    &z_{t | t}=z_{t | t-1}+M_{t}^{\text{opt}} \Delta y_{t} \\
    &\Sigma_{t | t}=\Sigma_{t | t-1}-M_{t}^{\text{opt}} H_{t} \Sigma_{t | t-1}
  \end{split}
  \]
  where ${M_{t}^{\text {opt}}=\Sigma_{t | t-1} H_{t}^{T}\left(H_{t} \Sigma_{t | t-1} H_{t}^{T}+R_{t}\right)^{-1}}$.
\end{enumerate}

\subsection{Extended KF}
The main idea behind the extension is to apply the standard KF in nonlinear systems. Let us consider a system in the form
\begin{equation}
  \begin{split}
    &x_{t+1}=f(x_t) + \xi_t\\
    &y_t=H_tx_t+\zeta_t\\
    &x_0\in\R^n, \ t=0,1,\dots
  \end{split}
\end{equation}
where $f:\R^n\rightarrow\R^n$ is a given vector function. In order to apply the KF, we wish to linearize the function $f(\cdot)$ around a reference point $x^{{\rm ref}}_t$, following the common linearization scheme using the jacobian of $f(\cdot)$.

Let
\[
\tilde{F}_t:=\dfrac{\partial f(x)}{\partial x}\bigg|_{x=x^{{\rm ref}}_t}
\]
with $\tilde{F}_t$ model (\ref{eq:kf_so}) can be constructed and applied to the KF.


\section{Instrumental Extended KF}\label{sec:IEKF}
In this work we will use a simple ARMA model, with correlated state and one unknown stationary parameter $a$. The model in consideration is
\begin{equation}
  \begin{split}
    &z_{t+1}=az_t+\xi_t+d\xi_{t-1}\\
    &y_{t}=hz_t+\gamma_{t}\\
    &z_0\in\R, \ t=0,1,\dots
  \end{split}
\end{equation}
and the objective is find an estimator $\tilde{a}$ for the unknown parameter $a$ using the data vector $y^*_t$.

First, we assume that $\tilde{a}$ is not a constant, even when the real parameter $a$ is stationary. This assumption takes into consideration that the data vector can be constantly changing and the estimation will change depending on each observation. Another important assumption is that this estimator $\tilde{a}_t$ has small linear changes, hence, this estimator takes the form
\begin{equation}
  \tilde{a}_t=b\tilde{a}_{t-1}+\chi_{t-1}
\end{equation}
with $0<b<1$ and $\chi_t \sim N(0,\,\sigma)$. Given the dynamics for $\tilde{a}_t$, we propose the complete dynamics for the ARMA model as
\begin{equation}\label{eq:mainARMA}
  \begin{split}
    &z_{t+1}=\tilde{a}_tz_t+\xi_t+d\xi_{t-1}\\
    &\tilde{a}_{t+1}=b\tilde{a}_t+\chi_t\\
    &z_0\in\R,\,\tilde{a}_0\in\R,\,t=0,1,\dots
  \end{split}
\end{equation}
Note that the dynamics have a nonlinear, particularly bilinear, term in the first equation. Let $x_t:=(z_t,\,\tilde{a}_t)^T$ and $x^{{\rm ref}}_t:=(z_t^{{\rm ref}},\,\tilde{a}_t^{{\rm ref}})^T$ be a given reference point, the linearization yields the matrix
\[
\tilde{F}_t = \begin{bmatrix}
\tilde{a} & z\\
0 & b
\end{bmatrix}_{x=x_t^{{\rm ref}}}
\]
Initially, the standard KF scheme could be applied, but one of the assumptions that is used to develop the KF procedure is that the state $z_t$ cannot be correlated with $z_{t-s}$ $\forall s$; clearly, $z_t$ and $z_{t-1}$ are correlated due to the delay in $\xi_t$ therefore, the state $x_t$ is correlated with $x_{t-1}$.

Taking into consideration these facts, the standard procedure cannot assure a ``adequate'' estimation of the state $z_t$. In order to address this problem, we suggeest an implementation with instrumental variables.

Let $\tilde{x}_t:=x_{2t}$, these transformation assures that the new state $\tilde{x}_t$ is not correlated with $\tilde{x}_{t-1}$; evidently, this transformation implies a new sampling in the available data, namely, we need to define a new data vector
 \[\tilde{y}_t^*:=\begin{cases}
   (y_0^T, y_2^T,\ldots,y_t^T)^T,&t\text{ even}\\
   (y_0^T, y_2^T,\ldots,y_{t-1}^T)^T,&t\text{ odd}
 \end{cases}\]
Now, given the instrumental variables transformation and the linearization procedure for model (\ref{eq:mainARMA}), the standard KF procedure can be applied.
