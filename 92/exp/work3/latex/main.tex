\documentclass[conference]{IEEEtran}
\IEEEoverridecommandlockouts
% The preceding line is only needed to identify funding in the first footnote. If that is unneeded, please comment it out.
\usepackage[utf8]{inputenc}
\usepackage[english]{babel}
\usepackage{cite}
\usepackage{amsmath,amssymb,amsfonts}
\usepackage{physics}
\usepackage{dsfont}
\usepackage{algorithmic}
\usepackage{graphicx}
\usepackage{subcaption}
\usepackage{array}
\usepackage{float}
\usepackage{lipsum}
\usepackage{textcomp}
\usepackage{xcolor}
\usepackage{diagbox}

\usepackage[hidelinks]{hyperref}
\def\BibTeX{{\rm B\kern-.05em{\sc i\kern-.025em b}\kern-.08em
    T\kern-.1667em\lower.7ex\hbox{E}\kern-.125emX}}



    \newtheorem{theorem}{Theorem}
    \newtheorem{corollary}{Corollary}
    \newtheorem{lemma}{Lemma}
    \newtheorem{definition}{Definition}
    \newtheorem{remark}{Remark}
    \newtheorem{example}{Example}
    %
    \newcommand{\N}{{\mathbb{N}}}
    \newcommand{\LL}{{\mathbb{L}}}
    \newcommand{\Z}{{\mathbb{Z}}}
    \newcommand{\Q}{{\mathbb{Q}}}
    \newcommand{\R}{{\mathbb{R}}}
    \newcommand{\W}{{\mathbb{W}}}
    \newcommand{\AC}{{\mathbb{AC}}}
    \newcommand{\C}{{\mathbb{C}}}
    \newcommand{\HH}{{\mathbb{H}}}
    \newcommand{\F}{{\mathbb{F}}}
    \newcommand{\PP}{{\mathbb{P}}}
    %
    \newcommand{\qed}{\hspace*{\fill}$\Box$}
    \newcommand{\prf}{\noindent{\it {Proof.}}\ }

    % INSERTE LINK  PUES PRRO
    \newcommand{\shorteq}{%
    \settowidth{\@tempdima}{-}% Width of hyphen
    \resizebox{\@tempdima}{\height}{=}%
    }

\begin{document}

\title{Kalman Filter for Observer-ARMA Model with Parameter Estimation\\}

\author{\IEEEauthorblockN{Juan Sebastián C\'ardenas-Rodríguez}
\IEEEauthorblockA{\textit{Mathematical Engineering} \\
\textit{Universidad EAFIT}\\
Medell\'in, Colombia\\
jscardenar@eafit.edu.co}
\and
\IEEEauthorblockN{David Plazas Escudero}
\IEEEauthorblockA{\textit{Mathematical Engineering} \\
\textit{Universidad EAFIT}\\
Medell\'in, Colombia \\
dplazas@eafit.edu.co}
}

\maketitle

\begin{abstract}
In this work we suggest a novel extension of the standard Kalman-Filter (KF), applied to parameter estimation and output-ARMA models. The procedure includes an extended KF for the bilinear form whilst estimating parameters and applying instrumental variables to eliminate state correlation, which can be reduced to a re-sample of the available data and apply the known KF. We include a short discussion regarding this method and some future work suggestions.
\end{abstract}

\begin{IEEEkeywords}
  State and parameter estimation, Kalman-Filter, instrumental variables, ARMA, extended Kalman-Filter.
\end{IEEEkeywords}
\section{Introduction}\label{sec:intro}
Autoregressive Moving-Average (ARMA) models are widely used in areas such as economics and finance, with applications in modelling time series. The main problems with ARMA models is the estimation of its parameters, using only state data. One important remark is that ARMA models do not consider an observer, which cannot be always assumed in real conditions.

Classical parameter estimation of ARMA models is often addressed with standard statistical methods, for instance, instrumental least-square (ILS) method. The main problem with the ILS method is the requirement of the complete state of the real system, which is not always achievable since the system is perceived only through the observer.


This work is focused on the design of a Kalman-based filter for the estimation of ARMA parameters through the output of the observer, taking into consideration that the disturbances in the ARMA models can be correlated and this violates one of the main hypothesis of the standard Kalman-Filter (KF). We propose in this work a filter based on instrumental variables and extended KF.

\section{Results}\label{sec:pre}
The real parameters are $a=(0.1, -0.2, 0.2)^T$, with initial conditions $z_0=(0,0,0,-0.1, -0.4,  0)^T$. In \cref{tab:params}, we present three different set of simulation parameters that were used, and the results are presented in \cref{fig:orig,fig:estimation,fig:estimation2}.


\begin{table}[H]
\centering
\begin{tabular}{llll}
\hline
\textbf{Set} & \textbf{\boldmath$C$} & \textbf{\boldmath$Q_t$} & \textbf{\boldmath$R_t$} \\ \hline
1            & $C_1$                 & $0.01 I_6$        & 25                      \\
2            & $C_2$                 & $I_6$                   & 5                       \\
3            & $C_3$                 & $I_6$                   & $5 I_3$           \\ \hline
\end{tabular}
\caption{Parameters for estimation procedure.}
\label{tab:params}
\end{table}
where
\[
\begin{split}
C_1&=\begin{bmatrix}
  1 & 0 & 0 & 0 & 0 & 0
\end{bmatrix}\\
C_2&=\begin{bmatrix}
  1 & 1 & 1 & 0 & 0 & 0
\end{bmatrix}\\
C_3&=\begin{bmatrix}
  1 & 0 & 0 & 0 & 0 & 0\\
  0 & 1 & 0 & 0 & 0 & 0\\
  0 & 0 & 1 & 0 & 0 & 0\\
\end{bmatrix}.
\end{split}
\]

\begin{figure}[H]
  \centering
  \includegraphics[scale=.4]{files/chinese_original.pdf}
  \caption{Parameter estimation using EKF for set 1.}
  \label{fig:orig}
\end{figure}

\begin{figure}[H]
  \centering
  \includegraphics[scale=.4]{files/chinese_estimation.pdf}
  \caption{Parameter estimation using EKF for set 2.}
  \label{fig:estimation}
\end{figure}

\begin{figure}
  \centering
  \includegraphics[scale=.4]{files/chinese_estimation_2.pdf}
  \caption{Parameter estimation using EKF for set 3.}
  \label{fig:estimation2}
\end{figure}

The results for set 1 (\cref{fig:orig}) show that the estimation does not present significant dynamics, therefore the parameters do not converge to the real values. This possible occurs since $Q_t$ is small; after testing with more scattered noises, we obtained more movement only in the parameter $a_1$, which is associated with the first equation that is also the output.


Attempting to solve this problem, we propose the set 2 and 3, on which the output includes all the original states. Consequently, more changes in the estimated parameters were obtained, but it did not converge to the desired value.

\section{Conclusions}\label{sec:conc}
All the sets proved that the proposed linearization is not suitable for estimating the real parameters. It may be possible that there exists a combination of simulation parameters such that the estimation converges, but it was not found in this work. This is left for future research.


\bibliography{bib}
\bibliographystyle{IEEEtran}
\end{document}
