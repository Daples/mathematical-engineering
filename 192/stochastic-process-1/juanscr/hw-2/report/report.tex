\documentclass[11pt]{article}
\usepackage[british]{babel}
\usepackage{geometry}
\usepackage{amsmath}
\usepackage{amsthm}
\usepackage{graphicx}
\usepackage[utf8]{inputenc}

%%%%%%%% HYPERREF PACKAGE
\usepackage{hyperref}
\hypersetup{citecolor=blue}
\hypersetup{urlcolor=blue}
\hypersetup{linkcolor=blue}
\hypersetup{colorlinks=true}

%%%%%%%% DEFINITION AND THEOREM DEFINITIONS
\theoremstyle{definition}
\newtheorem{definition}{Definición}[section]

\theoremstyle{remark}
\newtheorem{remark}{Apunte}

\theoremstyle{remark}
\newtheorem{question}{Pregunta}

\theoremstyle{theorem}
\newtheorem{theorem}{Teorema}

%%%%%%%% MULTI-COLUMNS PACKAGE
\usepackage{multicol}

%%%%%%%% BIB-LATEX STUFF
\usepackage[style=numeric,
            bibstyle=numeric,
            citestyle=numeric,
            hyperref=true,
            backend=biber]{biblatex}
\addbibresource{} %Put relative path to ref

%%%%%%%% START DOCUMENT

\title{Procesos Estocásticos}
\author{Juan Sebasti\'an C\'ardenas-Rodríguez\\ Código 201710008101 \\ \scalebox{0.7}{Ingeniería Matemática, Universidad EAFIT}}
\date{\today}


\begin{document}
\maketitle
\section{Problema}
En la data se registró la hora que fueron atendidos 106 pacientes de
un hospital. ¿Existe evidencia suficiente para decir que los pacientes
llegan según un proceso de Poisson? ¿Cuál es la intensidad del
proceso?.

\section{Solución}
Dado que tenemos los datos en que cada uno de los pacientes llega
podemos calcular el tiempo entre llegadas de cada uno de los
clientes. Así, podríamos evaluar si estos tiempos se distribuyen
exponencial; lo cual, si asumimos independencia por el teorema
\ref{theo:1} podemos demostrar que estas llegadas siguen un proceso de
poisson.

\begin{theorem}
  \label{theo:1}
  Sean $t_1$, $t_2$, $\dots$ variables aleatorias i.i.d. que
  distribuyen $\text{Exp}(\lambda)$. Entonces ocurre que, con
  $T_n = \sum_{i=1}^nt_i$, $\text{N}(s)= \max\{n: T_n \le s\}$ y
  $T_0 = 0$:

  \begin{itemize}
  \item $\text{N}(0) = 0$
  \item
    $\text{N}(t' + s) - \text{N}(t') \sim \text{Poisson}(\lambda s)$
  \item $\text{N}(s)$ tiene incrementos independientes.
  \end{itemize}
\end{theorem}

Como en este caso, los datos comienzan cuando el primer paciente
llega, entonces se debe suponer una hora inicial donde el sistema esté
vacío para cumplir con la condición que el sistema empieza sin ninguna
persona. Lo cual, para este trabajo se ausmió que el sistema comienza
a las 12:00:00 a.m. dado que la primera persona llegó 55 segundos
después.

En primer lugar, es importante aclarar que se dice que si una variable
aleatoria distribuye Exp$(\lambda)$ entonces es porque su función de
densidad está dada por:
\begin{equation*}
  f(x) =
  \begin{cases}
    \lambda e^{-\lambda x}, \text{ si } x \ge 0 \\
    0, \text{ en otro caso.}
  \end{cases}
\end{equation*}

Así, para observar si los tiempos entre llegadas distribuye
exponencial debemos realizar el test de Kolmogorov-Smirnov. Este test,
confirma si hay evidencia o no para rechazar que los datos distribuyen
una cierta distribución. Los tiempos entre llegadas, fueron calculados en minutos. Así, seleccionando el $\lambda$ así:

\begin{equation*}
  \lambda = \frac{106}{D} \approx 0.075
\end{equation*}

donde $D$ es el tiempo transcurrido entre el timepo inicial
seleccionado y el tiempo en que llegó la última persona al sistema.

El test hecho en el lenguaje de programación \textit{Python} 3.7.4,
con la librería \textit{SciPy}. Así se realizó el test con la
distribución exponencial, con su parametro seleccionado como explicado
anteriormente. El valor $p$ para este test es de:

\begin{equation*}
  p = 0.89
\end{equation*}

El nivel de significancia de la prueba fue seleccionado como
$\alpha = 0.05$. Lo cual, como $p > \alpha$, no hay evidencia para
rechazar que los datos distribuyen exponencial con la tasa dada. Así,
la intensidad de este proceso es el~$\lambda$ calculado, el cual sería
en horas de:

\begin{equation}
  \lambda \approx 4.5 \quad \frac{\text{pacientes}}{\text{hora}}
\end{equation}

Lo cual, por el teorema se confirma que es un proceso de Poisson.
\end{document}
