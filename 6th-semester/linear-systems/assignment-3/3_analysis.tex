\section{Results analysis}\label{sec:resultAn}
%%%%%%%%%%%%%%%%%%%%%%%%%%%%%%%%%%%%%%%%%%%%%%%%%%%%%%%%%%%%%%%
\subsection{Discrete PID Controller}
\subsubsection{Reaction curve method}
This method did not work neither using the Ziegler-Nichols nor the Chien-Hrones-Reswick rules as seen in Figs. \ref{fig:zeigler_linear}-\ref{fig:chr}. Although the value for $RL$, in both cases, is within the zone where the system is highly controllable there is an underlying problem within: the proportional gain gets larger as the value for the $RL$ becomes smaller. In this manner, in \cite{JS_PL2} it was found that the maximum gain for the closed-loop Rössler system was $k_{max}=601.2835$ for preserving stability and, the proportional gains for each of the methods provides a higher value. 
Furthermore, the Rössler system is a non-minimum phase attractor so when obtaining the line that crosses the inflection point it is affected by this property. Lastly, even if the problems explained above did not affect this controller, the parameters for the PID controller are large enough to always saturate the system.

Therefore, this heuristic fails as the proportional constant is higher than the one that the system can manage, the inflection point is not easily found with a non-minimum phase system and it does a large control action saturating the system.

\subsubsection{Sensitivity method}
As it is seen in Fig. \ref{fig:sens}, this controller completely reduces the $e_{ss}$ with an input of 1. This, different from the last PID, has a much better result as it takes into account the maximum gain obtained in the discrete system, assuring that the controller does not have undesired properties. Now, in Fig. \ref{fig:sens_lin_nonlin} it is seen how this controller acts perfectly on the linear system, as it should, while the Rössler system it stops working around the $u=1.6$; this is an expected behaviour, as the linear system is only a representation of the original one and, at the same time, the real system has a saturation which does not allow for unrealistically large inputs. On the other hand, around $u=-0.25$ it is found the lower bound for this reference (Fig. \ref{fig:sens_ref_lower}, as it is critically stable and smaller values will unstabilize the system.

Furthermore, this controller also works for ramp inputs, although in a smaller interval for values of the slope. As seen in Figs. \ref{fig:sens_ref_0_01t} and \ref{fig:sens_ref_-0_01t}, this controller effectively controls the system to have exactly the input given; although it does not match perfectly the input, this occurs because the PID controller is discrete therefore it can only act upon the system every time sample. It is important to notice, that if the system would be simulated more time this controller would not work as the input would surpass the maximum value found before; in this manner, the interval maximum and minimum values found here are dependent of the time of simulation. In this manner, in Figs. \ref{fig:sens_ref_0_018t}, \ref{fig:sens_ref_-0_03t} it is shown the interval in which the controller can successfully reduce the $e_{ss}$ for a ramp input. Lastly, it is important to notice that the controller can turn the system to a chaotic behaviour as shown in Fig. \ref{fig:sens_ref_-0_05t} due to the saturation of the system.

In conclusion, the discrete PID controller design using the sensitivity rules effectively controls the system for a step between $u \in [-0.25, 1.6]V$ and for a ramp input with a slope between $m \in [-0.03, 0.018]V/s$ approximately. 

\subsubsection{Analytic design}
In this design, it is important to notice that even though the controller was designed with a reduced system which did not represented the system in a transitory state, as shown in \cite{JS_PL2}, it accurately worked for designing this controller as shown in Figs. \ref{fig:anal_u_0_5}; this occurs because, this approximated system has some good properties that the original system does not possess but with some resemblance within the two. As shown in Fig. \ref{fig:anal_u_1_5}, the control action is barely not saturated as it is not passing the maximum value of the input 1500; in this manner, it is argued this is the upper bound for this controller as for more values it will start to saturate the system and not having a real action over it. On the other hand, for Fig. \ref{fig:anal_u_-1_2} it seen the lower bound for this input, as for smaller values the system is saturated and the controller does not stabilize the system.

Lastly, the lower bound for the poles available are approximately 0.5 because, as shown in \ref{fig:anal_u_0_5_poles_0_4}, if there are chosen a smaller poles than this value the controller cannot effectively stabilize the system and reduce the $e_{ss}$. This is easily explained the 0.4 poles are closer to the origin requiring a larger effort by the controller; in this manner, this effort saturates the system.

In conclusion, it was seen that the approximated system is useful to design a PID analytic controller with an interval for a step input of $u \in [-1.2, 1.5]V$ and for a chosen poles in $poles \in [0.5, 1)$. It is important to see that this controller has more range on the inputs that the design made by sensitivity.

%%%%%%%%%%%%%%%%%%%%%%%%%%%%%%%%%%%%%%%%%%
\subsection{Discrete state feedback controllers}
\subsubsection{$r(t) = 0$}
This design made with the Ackerman method successfully controlled the system as shown in Fig. \ref{fig:feedback_ref0_x30_1}. This is expected to work very efficiently, as it was selected to be a dead beat controller therefore being the more fast possible control. The upper bound for a change in the initial conditions of the output was shown in Fig. \ref{fig:feedback_ref0_x30_5} in which it is seen that the controller barely saturates the system; therefore, if this input would be larger the input would saturate the Rössler system. On the other hand, the lower bound for the initial condition of the output is $0$, as it is a positive voltage in the system; in this manner, as shown in Fig. \ref{fig:feedback_ref0_x3-0_26} it efficiently stabilizes the system in this lower bound. 

In conclusion, it was found that the controller successfully stabilizes correctly the Rössler system for changes in the initial condition of the output for $\Delta x_{3}(0) \in [-2.6, 50]V$.

\subsubsection{$r(t) \neq 0$}
This state feedback controller was successfully designed to reduced the $e_{ss}$ without saturating the system as shown in Fig. \ref{fig:feedback_u_1}. In this case, it was expected for this controller to work as the selected poles where the same as the ones chosen in the discrete PID analytic controller; on the other hand, it similarly occurs that smaller poles do not control the system as it saturates the system. Continuing with this similarity, the upper bound is the same one as show in Fig. \ref{fig:feedback_u_1_5} for the same reason. On the other hand, the lower bound found in Fig. \ref{fig:feedback_u_-2} is smaller, therefore being a more accurate controller; this occurs because, the feedback controller has access to more information than the PID making it possible to control smaller inputs in this manner. 

Lastly, in \ref{fig:feedback_u_ramp_0_01} it is seen that this controller, although it is capable to stabilize the system in a ramp, does not reduce the $e_{ss}$. This occurs because only one integrator was added to the system, therefore the $e_{ss}$ for an input ramp is a constant value. 

In conclusion, the state feedback discrete controller was successfully designed with an interval for an input of $u \in [-2, 1.5]V$. Furthermore, it can not reduce the $e_{ss}$ for a ramp as it was only added one integrator to the plant.

An important remark, is that for every controller explained in this section the interval for the inputs are really close to the operation point; this happens because, the Rössler system has a chaotic behaviour for different parameters and inputs making it difficult to control.

\subsection{Uncertainty Analysis}
\subsubsection{Analytic Discrete PID Controller}
As seen in Figs. \ref{fig:sens_ra_450} and \ref{fig:sens_ra_550} for a 10\% increase and decrease of the resistance $R_a$ this controller successfully controls the system and reduces completely the stationary error without saturating the system. Then, the upper bound for the resistance $a$, as shown in Fig. \ref{fig:sens_ra_2500} it is really high as for this value it barely saturates the system; in this manner, it is important to remark that the PID controller is highly robust as a controller designed for a value of a parameter works for a large margin of other values. Furthermore, in Fig. \ref{fig:sens_ra_300} a lower bound is found as for this value this controller makes it critically stable; this occurs because as found in \cite{JS_PL1} and \cite{JS_PL2}, for smaller values of this resistance the system starts behaving chaotically making it difficult to control it.

In conclusion, the discrete PID controller is a very robust controller as it can reduce the $e_{ss}$ for different margin of values for the $R_a \in [300, 2500] k\Omega$. 

\subsubsection{State Feedback Controller $r(t) = 0$}
This controller does not work well for changes in the $R_a$, as shown in Figs. \ref{fig:sens_ra_450_state} and \ref{fig:sens_ra_550_state}. This happens because this type of controller is only designed to work for this set of parameters, as this vector is the only one that is responsible for the controller; on the other hand, other controllers have different mechanisms that work together to ensure a robust control of the system. In this manner, this controller as it is such a minimalist one does not ensure control for similar systems varying the parameters. On the other hand, it is seen that this controller does stabilize the system in a quick way but not in 0, as it is desired.

In conclusion, this controller only stabilizes the system in 0 for the initial value of the resistance $R_a$, making it a not-robust controller different from the PID. It is important to notice that even if it uses more information than the PID, this controller has worse properties; so, the benefits of this controller are more on the side of having changes in the initial conditions instead of other values.


\subsubsection{State Feedback Controller $r(t) \neq 0$}
As presented in Figs. \ref{fig:sens_ra_550_state_noess}-\ref{fig:sens_ra_450_state_noess} (changes of 10\%, this controller behaves properly, eliminating steady-state error for the predefined reference $r(t)=1$. It is important to mention that this controller can still work really well for $R_a$ around $3000k\Omega$ (as Fig. \ref{fig:sens_ra_3000_state_noess}), where the system responds with no steady-state error but showing a negative overshoot. Note that for significant a larger change downwards ($R_a=250k\Omega$, Fig. \ref{fig:sens_ra_250_state_noess}), the system shows undesired behavior and cannot be accounted for a correct controller. In conclusion, just as the discrete PID controller designed, the state feedback controller shows robust behavior, working properly for resistors $R_a\in[450,3000]k\Omega$.