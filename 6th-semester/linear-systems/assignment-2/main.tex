\documentclass[conference]{IEEEtran}
\IEEEoverridecommandlockouts
% The preceding line is only needed to identify funding in the first footnote. If that is unneeded, please comment it out.
\usepackage[utf8]{inputenc}
\usepackage[english]{babel}
\usepackage{cite}
\usepackage{amsmath,amssymb,amsfonts}
\usepackage{physics}
\usepackage{dsfont}
\usepackage{algorithmic}
\usepackage{graphicx}
\usepackage{subcaption}
\usepackage{array}
\usepackage{float}
\usepackage{lipsum}
\usepackage{textcomp}
\usepackage{xcolor}

\usepackage[hidelinks]{hyperref}
\def\BibTeX{{\rm B\kern-.05em{\sc i\kern-.025em b}\kern-.08em
    T\kern-.1667em\lower.7ex\hbox{E}\kern-.125emX}}
\begin{document}

\title{Linear Analysis of Rössler System based on Circuits\\}

\author{\IEEEauthorblockN{Juan S. C\'ardenas R.}
\IEEEauthorblockA{\textit{Student} \\
\textit{Universidad EAFIT}\\
Medell\'in, Colombia\\
jscardenar@eafit.edu.co}
\and
\IEEEauthorblockN{David Plazas E.}
\IEEEauthorblockA{\textit{Student} \\
\textit{Universidad EAFIT}\\
Medell\'in, Colombia \\
dplazas@eafit.edu.co}
}

\maketitle

\begin{abstract}
In this work, the Rössler system is studied: a short introduction about the history of the system is given, as well as some state-of-the-art applications; a circuit implementation is presented based on the literature and translated to a state equation. The first part of this document is focused on give some basic theory, concepts and procedures to make a successful analysis of a linear system, which are applied and presented in the results section: a preliminary validation of the linear system obtained is performed making a comparison with the original Rössler system and an stability analysis is performed based on transfer functions and Bode diagrams. The simulations and the findings showed that the linear system in the selected operation point is stable, likewise intervals for one of the parameters that make the nonlinear and linear model stable as well; furthermore, the closed-loop stability was analyzed and properly validated. A linear approximation to Rössler equations was found with success (under some conditions).
\end{abstract}

\begin{IEEEkeywords}
Rössler system, simulation, state equation, dynamic system, transfer function, Bode diagram, stability analysis, linearization, order reduction, discrete dynamics, non-minimum phase system.
\end{IEEEkeywords}

\section{Introduction}\label{sec:intro}
Autoregressive Moving-Average (ARMA) models are widely used in areas such as economics and finance, with applications in modelling time series. The main problems with ARMA models is the estimation of its parameters, using only state data. One important remark is that ARMA models do not consider an observer, which cannot be always assumed in real conditions.

Classical parameter estimation of ARMA models is often addressed with standard statistical methods, for instance, instrumental least-square (ILS) method. The main problem with the ILS method is the requirement of the complete state of the real system, which is not always achievable since the system is perceived only through the observer.


This work is focused on the design of a Kalman-based filter for the estimation of ARMA parameters through the output of the observer, taking into consideration that the disturbances in the ARMA models can be correlated and this violates one of the main hypothesis of the standard Kalman-Filter (KF). We propose in this work a filter based on instrumental variables and extended KF.

\section{Results}\label{sec:results}
Remark: all the time-responses of the nonlinear Rössler system are presented subtracting the operation point, in order to see the references and stabilization easily.
\subsection{Discrete PID Controller}
\subsubsection{Tuning by Reaction Curve}
As described in section \ref{sec:tuning}, the tuning of a PID controller can be performed through the reaction curve. For obtaining the reaction curve, the linearized Rössler system (equation \ref{eq:LinearizedModel}) was simulated for $50s$ using $4^{th}$-order Runge-Kutta method with time step of $0.01s$, with input of $\Delta\mathbf{u}=2V$. The reaction curve obtained is presented in Fig. \ref{fig:reaction_curveRossler}.

\begin{figure}[H]
    \centering
    \includegraphics[scale=0.5]{files/reaction_curveRossler.pdf}
    \caption{Reaction curve.}
    \label{fig:reaction_curveRossler}
\end{figure}

Calculating the slope of this line and the delay: $R=0.0019$ and $L=0.8239$. This values yield the following PID parameters:
\begin{equation*}
    \begin{split}
        K_p&=782.8568\\
        T_i&=1.6478\\
        T_d&=0.4119
    \end{split}
\end{equation*}
Applying the formulas for the discrete PID, yields
\begin{equation*}
    \begin{split}
        q_0=&1342.9\\
        q_1=&-1190.3\\
        q_2=&322.4894
    \end{split}
\end{equation*}
Therefore, the transfer function of the discrete PID controller is
\begin{equation}
    \dfrac{U(z)}{E(z)}=\dfrac{1342.9z^2-1190.3z+322.4894}{z(z-1)}
\end{equation}

Now, a simulation applying this controller was performed using as reference $r(t)=1V$, for both the linear and nonlinear systems. In Fig. \ref{fig:zeigler_linear}, the control action and the output of the linear system is presented; note that this PID controller makes the linear system unstable, which is not desired.

    \begin{figure}
        \centering
        \begin{subfigure}[b]{0.475\textwidth}
            \centering
            \includegraphics[scale=0.425]{files/heuristic/zeigler/plot_control_zeigler_linear.pdf}
            \caption{Control action.}
        \end{subfigure}
        \vskip0.1cm
        \begin{subfigure}[b]{0.475\textwidth}   
            \centering 
            \includegraphics[scale=0.425]{files/heuristic/zeigler/plot_y_zeigler_linear.pdf}
            \caption{Output.}
        \end{subfigure}
        \caption{Linear system simulation with discrete Zeigler-Nichols controller.}
        \label{fig:zeigler_linear}
	\end{figure}
	
	On the other hand, the control action and the output of the nonlinear system is shown in Fig. \ref{fig:zeigler}. It is important to highlight that a saturation was selected for the input, i.e. a minimum and maximum value for the control action was added. This saturation was selected from $0V$ to a maximum of $1500V$, since it is a reasonable interval from a power source as input; note that the control action shown in Fig. \ref{fig:zeigler} a. was extracted before the saturation.
	
    \begin{figure}
        \centering
        \begin{subfigure}[b]{0.475\textwidth}
            \centering
            \includegraphics[scale=0.425]{files/heuristic/zeigler/plot_control_zeigler.pdf}
            \caption{Control action.}
        \end{subfigure}
        \vskip0.1cm
        \begin{subfigure}[b]{0.475\textwidth}   
            \centering 
            \includegraphics[scale=0.425]{files/heuristic/zeigler/plot_y_zeigler.pdf}
            \caption{Output.}
        \end{subfigure}
        \caption{Rössler system simulation with discrete Zeigler-Nichols controller.}
        \label{fig:zeigler}
	\end{figure}
	
	Moreover, the tuning was performed with the reaction curve but applying the Chien-Hrones-Reswick rules for $0\%$ of maximum overshoot. The discrete PID controller is
	\begin{equation*}
	    \dfrac{U(z)}{E(z)}=\dfrac{990.9356z^2-891.9540z+214.4554}{z(z-1)}
	\end{equation*}
	And the same simulation was performed with $r(t)=1V$. The plots presented in Fig. \ref{fig:chr_linear} show the control action for the linear system applying the Chien-Hrones-Reswick controller. Note that, just as Zeigler-Nichols controller, it makes the linear system unstable.
	
	\begin{figure}
        \centering
        \begin{subfigure}[b]{0.475\textwidth}
            \centering
            \includegraphics[scale=0.425]{files/heuristic/CHR/plot_control_CHR_linear.pdf}
            \caption{Control action.}
        \end{subfigure}
        \vskip0.1cm
        \begin{subfigure}[b]{0.475\textwidth}   
            \centering 
            \includegraphics[scale=0.425]{files/heuristic/CHR/plot_y_CHR_linear.pdf}
            \caption{Output.}
        \end{subfigure}
        \caption{Linear system simulation with discrete Chien-Hrones-Reswick controller.}
        \label{fig:chr_linear}
	\end{figure}
    
    Finally, the control action and system output of the Rössler nonlinear equations in study are presented in Fig. \ref{fig:chr}. Again, there is saturation in the control action 
    
    \begin{figure}
        \centering
        \begin{subfigure}[b]{0.475\textwidth}
            \centering
            \includegraphics[scale=0.425]{files/heuristic/CHR/plot_control_CHR.pdf}
            \caption{Control action.}
        \end{subfigure}
        \vskip0.1cm
        \begin{subfigure}[b]{0.475\textwidth}   
            \centering 
            \includegraphics[scale=0.425]{files/heuristic/CHR/plot_y_CHR.pdf}
            \caption{Output.}
        \end{subfigure}
        \caption{Rössler system simulation with discrete Chien-Hrones-Reswick controller.}
        \label{fig:chr}
	\end{figure}


    \subsubsection{Tuning by Sensitivity Curve}
    The second method presented in \ref{sec:tuning} is through the gain margin $M_G$ and its frequency $\omega_{cf}$. In previous work \cite{JS_PL2}, the respective gain margin and frequency was obtained: 
    $M_G=52.7993dB$ and $\omega_{cf}=3.1416rad/s$, which yield $K_u=436.48$ and $T_u=2s$. Therefore, the discrete PID controller with the sensitivity method is
    \begin{equation*}
	    \dfrac{U(z)}{E(z)}=\dfrac{458.3046z^2-261.8883z+65.4721}{z(z-1)}
	\end{equation*}
    In order to make a comparison with the previously obtained PID controllers, the same simulation will be performed with $r(t)=1$. In Fig. \ref{fig:sens}, the control action and output for the nonlinear Rössler system is presented. Note that, unlike the previous PID controllers, this one does not unstabilize the system and can eliminate eliminate steady-state error for this particular reference.
    
    \begin{figure}
        \centering
        \begin{subfigure}[b]{0.475\textwidth}
            \centering
            \includegraphics[scale=0.425]{files/heuristic/Sensitivity/control_sens_u_1.pdf}
            \caption{Control action.}
        \end{subfigure}
        \vskip0.1cm
        \begin{subfigure}[b]{0.475\textwidth}   
            \centering 
            \includegraphics[scale=0.425]{files/heuristic/Sensitivity/sens_u_1.pdf}
            \caption{Output.}
        \end{subfigure}
        \caption{Rössler system simulation with discrete sensitivity controller.}
        \label{fig:sens}
	\end{figure}
	
    It is important to highlight that the linear system is not presented since it works perfectly with this controller. Note that the system shows stabilization in $1V$. Now that a functional controller was found, we proceed to make an analysis on this controller.
    
    In order to start the analysis, the following plots (Fig. \ref{fig:sens_lin_nonlin}) show the steady-state values for the output and the control signal for different inputs; this figure was constructed following the same ideas behind a linearity curve (see \cite{JS_PL2}). Note that the plots show that the nonlinear system diverges from the linear controlled system, due to the saturation defined in the system.
    
    \begin{figure}
        \centering
        \begin{subfigure}[b]{0.475\textwidth}
            \centering
            \includegraphics[scale=0.425]{files/heuristic/Sensitivity/control_sens_lin_vs_nonlin.pdf}
            \caption{Control action.}
        \end{subfigure}
        \vskip0.1cm
        \begin{subfigure}[b]{0.475\textwidth}
            \centering 
            \includegraphics[scale=0.425]{files/heuristic/Sensitivity/sens_lin_vs_nonlin.pdf}
            \caption{Output.}
        \end{subfigure}
        \caption{Sensitivity linear vs. nonlinear system.}
        \label{fig:sens_lin_nonlin}
	\end{figure}
	
    
    Now, it was desired test the controller with time-dependent references. The first simulation was for a reference $r(t)=0.01t$ with $t\in[0,100]$. The result of this simulation for both control and output is presented in Fig. \ref{fig:sens_ref_0_01t}; note that the control system makes a good attempt to stabilize the system exactly at the reference, making the error almost constant. Note that the control action remains inside the saturation interval, meaning that the control does not exceed the predefined limit.
    
    \begin{figure}
        \centering
        \begin{subfigure}[b]{0.475\textwidth}
            \centering
            \includegraphics[scale=0.425]{files/heuristic/Sensitivity/control_sens_ramp_ref_0_01.pdf}
            \caption{Control action.}
        \end{subfigure}
        \vskip0.1cm
        \begin{subfigure}[b]{0.475\textwidth}   
            \centering
            \includegraphics[scale=0.425]{files/heuristic/Sensitivity/sens_ramp_ref_0_01.pdf}
            \caption{Output.}
        \end{subfigure}
        \caption{Rössler system simulation with discrete sensitivity controller for $r(t)=0.01t$.}
        \label{fig:sens_ref_0_01t}
	\end{figure}
    
    Another simulation was executed but for a slightly larger slope: $r(t)=0.018$. The results of this simulation are displayed in Fig. \ref{fig:sens_ref_0_018t}. Note that the control action reaches its maximum value near $90s$, therefore there is saturation and the nonlinear system can no longer ``follow'' the reference, as the output shows: around $90s$ both curves diverge.
    
    \begin{figure}
        \centering
        \begin{subfigure}[b]{0.475\textwidth}
            \centering
            \includegraphics[scale=0.425]{files/heuristic/Sensitivity/control_sens_ramp_ref_0_018.pdf}
            \caption{Control action.}
        \end{subfigure}
        \vskip0.1cm
        \begin{subfigure}[b]{0.475\textwidth}   
            \centering 
            \includegraphics[scale=0.425]{files/heuristic/Sensitivity/sens_ramp_ref_0_018.pdf}
            \caption{Output.}
        \end{subfigure}
        \caption{Rössler system simulation with discrete sensitivity controller for $r(t)=0.018t$.}
        \label{fig:sens_ref_0_018t}
	\end{figure}
    
    As it will be discussed later on the Analysis section \ref{sec:resultAn}, this discrete PID controller has an upper bound. In order to find a lower bound, negative slope references will be analyzed. For a reference of $r(t)=-0.01$, Fig. \ref{fig:sens_ref_-0_01t} shows the control and output; just as its positive counterpart, the nonlinear ``follows'' the negative reference with almost a constant error.
    
    \begin{figure}
        \centering
        \begin{subfigure}[b]{0.475\textwidth}
            \centering
            \includegraphics[scale=0.425]{files/heuristic/Sensitivity/control_sens_ramp_ref_-0_01.pdf}
            \caption{Control action.}
        \end{subfigure}
        \vskip0.1cm
        \begin{subfigure}[b]{0.475\textwidth}   
            \centering 
            \includegraphics[scale=0.425]{files/heuristic/Sensitivity/sens_ramp_ref_-0_01.pdf}
            \caption{Output.}
        \end{subfigure}
        \caption{Rössler system simulation with discrete sensitivity controller for $r(t)=-0.01t$.}
        \label{fig:sens_ref_-0_01t}
	\end{figure}
	
	For a slightly more negative slope ($r(t)=-0.03t$), the plots presented in Fig. \ref{fig:sens_ref_-0_03t} show that the control action needed is negative, which is a contradiction to the minimum control action set, making the nonlinear system diverge from the reference.
	
	\begin{figure}
        \centering
        \begin{subfigure}[b]{0.475\textwidth}
            \centering
            \includegraphics[scale=0.425]{files/heuristic/Sensitivity/control_sens_ramp_ref_-0_03.pdf}
            \caption{Control action.}
        \end{subfigure}
        \vskip0.1cm
        \begin{subfigure}[b]{0.475\textwidth}   
            \centering
            \includegraphics[scale=0.425]{files/heuristic/Sensitivity/sens_ramp_ref_-0_03.pdf}
            \caption{Output.}
        \end{subfigure}
        \caption{Rössler system simulation with discrete sensitivity controller for $r(t)=-0.03t$.}
        \label{fig:sens_ref_-0_03t}
	\end{figure}
	
	For a better visualization of the effect of this lower saturation, one last simulation was performed with the discrete sensitivity PID controller with $r(t)=-0.05t$; Fig. \ref{fig:sens_ref_-0_05t} shows the results. Note that the control action is negative from around $50s$, at that exact time the nonlinear Rössler system diverges completely from the reference and starts to behave chaotically.
    
    \begin{figure}
        \centering
        \begin{subfigure}[b]{0.475\textwidth}
            \centering
            \includegraphics[scale=0.425]{files/heuristic/Sensitivity/control_sens_ramp_ref_-_0_05.pdf}
            \caption{Control action.}
        \end{subfigure}
        \vskip0.1cm
        \begin{subfigure}[b]{0.475\textwidth}   
            \centering 
            \includegraphics[scale=0.425]{files/heuristic/Sensitivity/sens_ramp_ref_-0_05.pdf}
            \caption{Output.}
        \end{subfigure}
        \caption{Rössler system simulation with discrete sensitivity controller for $r(t)=-0.05t$.}
        \label{fig:sens_ref_-0_05t}
	\end{figure}
    
    Finally, a lower bound for the reference was found by iterative simulation: $r(t)=-0.25$; Fig. \ref{fig:sens_ref_lower} shows the control action and output with the sensitivity discrete PID controller on its critically stable state.
    
    \begin{figure}
        \centering
        \begin{subfigure}[b]{0.475\textwidth}
            \centering
            \includegraphics[scale=0.425]{files/heuristic/Sensitivity/control_sens_low_bound_ref.pdf}
            \caption{Control action.}
        \end{subfigure}
        \vskip0.1cm
        \begin{subfigure}[b]{0.475\textwidth}   
            \centering 
            \includegraphics[scale=0.425]{files/heuristic/Sensitivity/sens_u_-0_25.pdf}
            \caption{Output.}
        \end{subfigure}
        \caption{Rössler system simulation with discrete sensitivity controller for $r(t)=-0.25t$.}
        \label{fig:sens_ref_lower}
	\end{figure}
    
    %%%%%%%%%%%%%%%%%%%%%%%
    \subsubsection{Tuning by Analytic Procedure}
    In previous work \cite{JS_PL2}, a second order approximation for the linearized Rössler system was found and has the following transfer function:
    \begin{equation}\label{eq:reduced_order2}
    \tilde{G}(s)=\dfrac{0.001461}{s^2 + 0.5887 s + 0.5102}
    \end{equation}
    and applying a discretization of this transfer function with sample time of $T=1s$, the following discrete system was obtained:
    \begin{equation}
    \tilde{G}(z) = \dfrac{0.000582z+0.0004769}{z^2-1.185z+0.5547}
    \end{equation}
    
    
    The analytic procedure was applied setting all four poles in $0.5$ (closed-loop system), which yields the following discrete PID controller
    \begin{equation}
    \dfrac{U(z)}{E(z)}=\dfrac{212.6018z^2-355.9319z+202.3534}{(z+0.0613)(z-1)}
    \end{equation}
    
    With the obtained discrete PID controller, a simulation was performed for a reference of $r(t)=0.5$ (Fig. \ref{fig:anal_u_0_5}), note that the output is fast and it does not have steady-state error but the response shows high frequency oscillations.
    \begin{figure}
        \centering
        \begin{subfigure}[b]{0.475\textwidth}
            \centering
            \includegraphics[scale=0.425]{files/heuristic/analytic/control_analytic_u_0_5.pdf}
            \caption{Control action.}
        \end{subfigure}
        \vskip0.1cm
        \begin{subfigure}[b]{0.475\textwidth}   
            \centering 
            \includegraphics[scale=0.425]{files/heuristic/analytic/analytic_u_0_5.pdf}
            \caption{Output.}
        \end{subfigure}
        \caption{Rössler system simulation with discrete analytic controller for $r(t)=0.5$.}
        \label{fig:anal_u_0_5}
	\end{figure}
	
	Next, a simulation for $r(t)=1.5$ was performed and presented in Fig. \ref{fig:anal_u_1_5}. Notice the high frequency oscillation in transitory state but the smooth stabilization exactly at $1.5$, eliminating the steady-state error.
	\begin{figure}
        \centering
        \begin{subfigure}[b]{0.475\textwidth}
            \centering
            \includegraphics[scale=0.425]{files/heuristic/analytic/control_analytic_u_1_5.pdf}
            \caption{Control action.}
        \end{subfigure}
        \vskip0.1cm
        \begin{subfigure}[b]{0.475\textwidth}   
            \centering 
            \includegraphics[scale=0.425]{files/heuristic/analytic/analytic_u_1_5.pdf}
            \caption{Output.}
        \end{subfigure}
        \caption{Rössler system simulation with discrete analytic controller for $r(t)=1.5$.}
        \label{fig:anal_u_1_5}
	\end{figure}
	
	Finally, another simulation was executed for a negative reference value of $r(t)=-1.2$. The control action and the system output are shown in Fig. \ref{fig:anal_u_-1_2}. This time response is quite fast and shows high frequency oscillation as well. The system does not present steady-state error.
	\begin{figure}
        \centering
        \begin{subfigure}[b]{0.475\textwidth}
            \centering
            \includegraphics[scale=0.425]{files/heuristic/analytic/control_analytic_u_-1_2.pdf}
            \caption{Control action.}
        \end{subfigure}
        \vskip0.1cm
        \begin{subfigure}[b]{0.475\textwidth}   
            \centering 
            \includegraphics[scale=0.425]{files/heuristic/analytic/analytic_u_-1_2.pdf}
            \caption{Output.}
        \end{subfigure}
        \caption{Rössler system simulation with discrete analytic controller for $r(t)=-1.2$.}
        \label{fig:anal_u_-1_2}
	\end{figure}
	
	Moreover, it was attempted to take the poles to $z=0.4$, which yields the following PID controller
	\begin{equation}
    \dfrac{U(z)}{E(z)}=\dfrac{547.6520z^2-788.6674z+363.4062}{(z+0.2663)(z-1)}
	\end{equation}
    In order to test this controller, a reference of $r(t)=0.5$ was used for the simulation presented in Fig. \ref{fig:anal_u_0_5_poles_0_4}; this simulation shows that the PID controller fails and does not behave properly.
	
	\begin{figure}
        \centering
        \begin{subfigure}[b]{0.475\textwidth}
            \centering
            \includegraphics[scale=0.425]{files/heuristic/analytic/control_analytic_u_0_5_poles_0_4.pdf}
            \caption{Control action.}
        \end{subfigure}
        \vskip0.1cm
        \begin{subfigure}[b]{0.475\textwidth}   
            \centering 
            \includegraphics[scale=0.425]{files/heuristic/analytic/analytic_u_0_5_poles_0_4.pdf}
            \caption{Output.}
        \end{subfigure}
        \caption{Rössler system simulation with discrete analytic controller for $r(t)=0.5$ with poles in $0.4$.}
        \label{fig:anal_u_0_5_poles_0_4}
	\end{figure}
	
	
	\subsection{Discrete State Feedback Controller: Pole Assignment}
	It was desired to set the poles exactly at the origin, in $z=0$ (dead-beat controller), with the state feedback control scheme. Applying the method described in \ref{sec:state_feed}, the obtained gain vector is
    \begin{equation}
        K=[-438.5251,\,-103.1074,\,86.1249]
    \end{equation}
	
	
	In order to test the obtained controller, some simulations were performed. Keep in mind that this method is only for $r(t)=0$, therefore the simulations are performed changing the initial conditions for the state $x_3$ (output). The first change is in $0.1V$ and the obtained results are presented in Fig. \ref{fig:feedback_ref0_x30_1}.
	\begin{figure}
        \centering
        \begin{subfigure}[b]{0.475\textwidth}
            \centering
            \includegraphics[scale=0.425]{files/feedback/Ref0/control_sfc_x30_1_ref_0.pdf}
            \caption{Control action.}
        \end{subfigure}
        \vskip0.1cm
        \begin{subfigure}[b]{0.475\textwidth}   
            \centering 
            \includegraphics[scale=0.425]{files/feedback/Ref0/sfc_x30_1_ref_0.pdf}
            \caption{Output.}
        \end{subfigure}
        \caption{Rössler system simulation with discrete state feedback controller with $0.1$ in $x_3$.}
        \label{fig:feedback_ref0_x30_1}
	\end{figure}
	
	Next, a simulation changing $50$ in the initial conditions of $x_3$ was carried out; in Fig. \ref{fig:feedback_ref0_x30_5} the control action and the system output can be appreciated. Note that the system can eliminate steady-state error but at the very beginning there is saturation.
	\begin{figure}
        \centering
        \begin{subfigure}[b]{0.475\textwidth}
            \centering
            \includegraphics[scale=0.425]{files/feedback/Ref0/control_sfc_x30_50_ref_0.pdf}
            \caption{Control action.}
        \end{subfigure}
        \vskip0.1cm
        \begin{subfigure}[b]{0.475\textwidth}   
            \centering 
            \includegraphics[scale=0.425]{files/feedback/Ref0/sfc_x30_50_ref_0.pdf}
            \caption{Output.}
        \end{subfigure}
        \caption{Rössler system simulation with discrete state feedback controller with $50$ in $x_3$.}
        \label{fig:feedback_ref0_x30_5}
	\end{figure}
	
	Now, testing for a negative change in the initial condition, it was subtracted $2.6$ from the initial conditions of $x_3$. In Fig. \ref{fig:feedback_ref0_x3-0_26} these results are presented for the control and the output. Note that there is no saturation neither steady-state error.
	\begin{figure}
        \centering
        \begin{subfigure}[b]{0.475\textwidth}
            \centering
            \includegraphics[scale=0.425]{files/feedback/Ref0/control_sfc_x30_-2_6_ref_0.pdf}
            \caption{Control action.}
        \end{subfigure}
        \vskip0.1cm
        \begin{subfigure}[b]{0.475\textwidth}   
            \centering 
            \includegraphics[scale=0.425]{files/feedback/Ref0/sfc_x30_-2_6_ref_0.pdf}
            \caption{Output.}
        \end{subfigure}
        \caption{Rössler system simulation with discrete state feedback controller with $-2.6$ in $x_3$.}
        \label{fig:feedback_ref0_x3-0_26}
	\end{figure}
	
    \subsection{Discrete State Feedback Controller: Pole Assignment without $e_{ss}$}
    In this section, all the desired poles for the system were $z=0.5$ (prior analysis yielded that $z=0$ required way too much energy). Carrying out the procedures depicted in section \ref{sec:state_feed_noess}, the gain vectors are
	\begin{equation}
	\begin{split}
	    K=&[-24.9703,\,156.6863,\,14.2350]\\
	    L=&-28.1628
	\end{split}
	\end{equation}
	
	It has been mentioned that this controller allows for $r(t)\neq0$; therefore, in order to test this controller and to compare with the previously obtained ones, a simulation was conducted using $r(t)=1$. The results for said simulation are shown in Fig. \ref{fig:feedback_u_1}. Note that this controller is achieves stability and can eliminate steady-state error for $r(t)=1$.
	
	\begin{figure}
        \centering
        \begin{subfigure}[b]{0.475\textwidth}
            \centering
            \includegraphics[scale=0.425]{files/feedback/Ref!0/control_sfc_u_1_ref_dif_0.pdf}
            \caption{Control action.}
        \end{subfigure}
        \vskip0.1cm
        \begin{subfigure}[b]{0.475\textwidth}   
            \centering 
            \includegraphics[scale=0.425]{files/feedback/Ref!0/sfc_u_1_ref_dif_0.pdf}
            \caption{Output.}
        \end{subfigure}
        \caption{Rössler system simulation with discrete state feedback controller $r(t)=1$.}
        \label{fig:feedback_u_1}
	\end{figure}
	
	Another simulation was performed using $r(t)=1.5$ (Fig. \ref{fig:feedback_u_1_5}) and the system showed proper behavior with this controller, eliminating steady-state error but with slightly larger overshoot without saturation.
	\begin{figure}
        \centering
        \begin{subfigure}[b]{0.475\textwidth}
            \centering
            \includegraphics[scale=0.425]{files/feedback/Ref!0/control_sfc_u_1_5_ref_dif_0.pdf}
            \caption{Control action.}
        \end{subfigure}
        \vskip0.1cm
        \begin{subfigure}[b]{0.475\textwidth}   
            \centering 
            \includegraphics[scale=0.425]{files/feedback/Ref!0/sfc_u_1_5_ref_dif_0.pdf}
            \caption{Output.}
        \end{subfigure}
        \caption{Rössler system simulation with discrete state feedback controller $r(t)=1.5$.}
        \label{fig:feedback_u_1_5}
	\end{figure}
	
	Finally, one last constant reference was simulated: $r(t)=-2$. As shown in Fig. \ref{fig:feedback_u_-2}, the controller manages to stabilize the system eliminating steady-state error but with high frequency oscillations.
	
	\begin{figure}
        \centering
        \begin{subfigure}[b]{0.475\textwidth}
            \centering
            \includegraphics[scale=0.425]{files/feedback/Ref!0/control_sfc_u_-2_ref_dif_0.pdf}
            \caption{Control action.}
        \end{subfigure}
        \vskip0.1cm
        \begin{subfigure}[b]{0.475\textwidth}   
            \centering 
            \includegraphics[scale=0.425]{files/feedback/Ref!0/sfc_u_-2_ref_dif_0.pdf}
            \caption{Output.}
        \end{subfigure}
        \caption{Rössler system simulation with discrete state feedback controller $r(t)=-2$.}
        \label{fig:feedback_u_-2}
	\end{figure}
	
	Next, a variable reference was tested, using $r(t)=0.01t$. The control action and the system's output is presented in \ref{fig:feedback_u_ramp_0_01}; notice that the steady-state for the Rössler system is parallel to the desired reference, this will be discussed in the next section.
	
	\begin{figure}
        \centering
        \begin{subfigure}[b]{0.475\textwidth}
            \centering
            \includegraphics[scale=0.425]{files/feedback/Ref!0/control_sfc_ramp_0_01_ref_dif_0.pdf}
            \caption{Control action.}
        \end{subfigure}
        \vskip0.1cm
        \begin{subfigure}[b]{0.475\textwidth}   
            \centering 
            \includegraphics[scale=0.425]{files/feedback/Ref!0/sfc_ramp_0_01_ref_dif_0.pdf}
            \caption{Output.}
        \end{subfigure}
        \caption{Rössler system simulation with discrete state feedback controller $r(t)=-2$.}
        \label{fig:feedback_u_ramp_0_01}
	\end{figure}
	
\subsection{Uncertainty Analysis}
The uncertainty analysis was performed for the analytic discrete PID controller, for the state feedback controller with and without steady-state error. Applying the procedure described in section \ref{sec:uncertainty}, the parameter $R_a=500k\Omega$ will be first changed by $10\%$ (upwards and downwards) and then some other important values will be selected. The following simulations are set for $r(t)=1$.

For the analytic discrete PID controller, the first two simulations are for changing $R_a$ by $10\%$ upwards ($550k\Omega$) and downwards ($450k\Omega$). In Figs. \ref{fig:sens_ra_550}-\ref{fig:sens_ra_450}, these simulations are presented, notice that this controller works properly for these changes in $R_a$.

    \begin{figure}
        \centering
        \begin{subfigure}[b]{0.475\textwidth}
            \centering
            \includegraphics[scale=0.425]{files/sens_analysis/PID/control_analytic_a_550.pdf}
            \caption{Control action.}
        \end{subfigure}
        \vskip0.1cm
        \begin{subfigure}[b]{0.475\textwidth}   
            \centering 
            \includegraphics[scale=0.425]{files/sens_analysis/PID/analytic_sensitivity_a_550.pdf}
            \caption{Output.}
        \end{subfigure}
        \caption{Analytic discrete PID controller for $R_a=550k\Omega$.}
        \label{fig:sens_ra_550}
	\end{figure}
	
	\begin{figure}
        \centering
        \begin{subfigure}[b]{0.475\textwidth}
            \centering
            \includegraphics[scale=0.425]{files/sens_analysis/PID/control_analytic_a_450.pdf}
            \caption{Control action.}
        \end{subfigure}
        \vskip0.1cm
        \begin{subfigure}[b]{0.475\textwidth}   
            \centering 
            \includegraphics[scale=0.425]{files/sens_analysis/PID/analytic_sensitivity_a_450.pdf}
            \caption{Output.}
        \end{subfigure}
        \caption{Analytic discrete PID controller for $R_a=450k\Omega$.}
        \label{fig:sens_ra_450}
	\end{figure}
	
	Next, a larger change upwards in $R_a$ was performed: $R_a=2500k\Omega$. In Fig. \ref{fig:sens_ra_2500} the results for the controller and the output are presented, note that the system achieves stability and steady-state error elimination with smaller overshoot than the previous simulations; but the control action is on the limit, as it is about to reach its maximum value.
	
	\begin{figure}
        \centering
        \begin{subfigure}[b]{0.475\textwidth}
            \centering
            \includegraphics[scale=0.425]{files/sens_analysis/PID/control_analytic_a_2500.pdf}
            \caption{Control action.}
        \end{subfigure}
        \vskip0.1cm
        \begin{subfigure}[b]{0.475\textwidth}   
            \centering 
            \includegraphics[scale=0.425]{files/sens_analysis/PID/analytic_sensitivity_a_2500.pdf}
            \caption{Output.}
        \end{subfigure}
        \caption{Analytic discrete PID controller for $R_a=2500k\Omega$.}
        \label{fig:sens_ra_2500}
	\end{figure}
	
	Finally, a larger change downwards was performed $R_a=300k\Omega$. Fig. \ref{fig:sens_ra_300} shows the results for this simulation. Notice that the system is critically stable around $y(t)=1V$, this implies that the controller is on the edge of making the system unstable and it cannot eliminate steady-steady error. Note that the control signal is almost saturated.
	\begin{figure}
        \centering
        \begin{subfigure}[b]{0.475\textwidth}
            \centering
            \includegraphics[scale=0.425]{files/sens_analysis/PID/control_analytic_a_300.pdf}
            \caption{Control action.}
        \end{subfigure}
        \vskip0.1cm
        \begin{subfigure}[b]{0.475\textwidth}   
            \centering 
            \includegraphics[scale=0.425]{files/sens_analysis/PID/analytic_sensitivity_a_300.pdf}
            \caption{Output.}
        \end{subfigure}
        \caption{Analytic discrete PID controller for $R_a=300k\Omega$.}
        \label{fig:sens_ra_300}
	\end{figure}
	
	For the discrete state feedback controller, the same procedure was attempted. The parameter $R_a$ was changed to $550k\Omega$ and $450k\Omega$. Figs. \ref{fig:sens_ra_550_state}-\ref{fig:sens_ra_450_state} show both simulations (recall that $r(t)=0$ in these controllers), where it can be observed that these controller does not work properly (they do not stabilize in the reference).
	
	\begin{figure}
        \centering
        \begin{subfigure}[b]{0.475\textwidth}
            \centering
            \includegraphics[scale=0.425]{files/sens_analysis/Ref0/control_analysis_sfc_a_550.pdf}
            \caption{Control action.}
        \end{subfigure}
        \vskip0.1cm
        \begin{subfigure}[b]{0.475\textwidth}   
            \centering 
            \includegraphics[scale=0.425]{files/sens_analysis/Ref0/analysis_sfc_a_550.pdf}            \caption{Output.}
        \end{subfigure}
        \caption{Discrete state feedback controller controller for $R_a=550k\Omega$.}
        \label{fig:sens_ra_550_state}
	\end{figure}
	
	\begin{figure}
        \centering
        \begin{subfigure}[b]{0.475\textwidth}
            \centering
            \includegraphics[scale=0.425]{files/sens_analysis/Ref0/control_analysis_sfc_a_450.pdf}
            \caption{Control action.}
        \end{subfigure}
        \vskip0.1cm
        \begin{subfigure}[b]{0.475\textwidth}   
            \centering 
            \includegraphics[scale=0.425]{files/sens_analysis/Ref0/analysis_sfc_a_450.pdf}
            \caption{Output.}
        \end{subfigure}
        \caption{Discrete state feedback controller controller for $R_a=450k\Omega$.}
        \label{fig:sens_ra_450_state}
	\end{figure}
	
	Finally, the discrete state feedback controller with no steady-state error was analyzed with the same parameters as the discrete analytical PID controller. Thus, the first two simulations are for $R_a=550k\Omega$ and $R_a=450k\Omega$, displayed in Figs. \ref{fig:sens_ra_550_state_noess}-\ref{fig:sens_ra_450_state_noess}, as it can be seen, the controller works properly for both changes, showing no steady-state error and good behavior (no significant overshoot or oscillation).
	
	\begin{figure}
        \centering
        \begin{subfigure}[b]{0.475\textwidth}
            \centering
            \includegraphics[scale=0.425]{files/sens_analysis/Ref!0/control_analysis_sfc_a_550_ref_dif_0.pdf}
            \caption{Control action.}
        \end{subfigure}
        \vskip0.1cm
        \begin{subfigure}[b]{0.475\textwidth}   
            \centering 
            \includegraphics[scale=0.425]{files/sens_analysis/Ref!0/analysis_sfc_a_550_ref_dif_0.pdf}            \caption{Output.}
        \end{subfigure}
        \caption{Discrete state feedback controller ($r(t)\neq0$) controller for $R_a=550k\Omega$.}
        \label{fig:sens_ra_550_state_noess}
	\end{figure}
	
	\begin{figure}
        \centering
        \begin{subfigure}[b]{0.475\textwidth}
            \centering
            \includegraphics[scale=0.425]{files/sens_analysis/Ref!0/control_analysis_sfc_a_450_ref_dif_0.pdf}
            \caption{Control action.}
        \end{subfigure}
        \vskip0.1cm
        \begin{subfigure}[b]{0.475\textwidth}   
            \centering 
            \includegraphics[scale=0.425]{files/sens_analysis/Ref!0/analysis_sfc_a_450_ref_dif_0.pdf}
            \caption{Output.}
        \end{subfigure}
        \caption{Discrete state feedback controller ($r(t)\neq0$) controller for $R_a=450k\Omega$.}
        \label{fig:sens_ra_450_state_noess}
	\end{figure}
	
    Now, the same simulation was executed for $R_a=3000k\omega$ and the control signal and output can be observed in Fig. \ref{fig:sens_ra_3000_state_noess}. Notice that the control signal is almost saturated but the controller works perfectly, without overshoot relatively fast stabilization.
    \begin{figure}
        \centering
        \begin{subfigure}[b]{0.475\textwidth}
            \centering
            \includegraphics[scale=0.425]{files/sens_analysis/Ref!0/control_analysis_sfc_a_3000_ref_dif_0.pdf}
            \caption{Control action.}
        \end{subfigure}
        \vskip0.1cm
        \begin{subfigure}[b]{0.475\textwidth}   
            \centering 
            \includegraphics[scale=0.425]{files/sens_analysis/Ref!0/analysis_sfc_a_3000_ref_dif_0.pdf}
            \caption{Output.}
        \end{subfigure}
        \caption{Discrete state feedback controller ($r(t)\neq0$) controller for $R_a=3000k\Omega$.}
        \label{fig:sens_ra_3000_state_noess}
	\end{figure}
	
	Lastly, the parameter $R_a$ was set to $250k\Omega$ and the simulation was conducted one last time. These results are presented in Fig. \ref{fig:sens_ra_250_state_noess}; this simulation show that the controller cannot work properly, since the system reaches a critically-stable state, showing undesired steady-state error (although the control signal is not saturated).
	
	\begin{figure}
        \centering
        \begin{subfigure}[b]{0.475\textwidth}
            \centering
            \includegraphics[scale=0.425]{files/sens_analysis/Ref!0/control_analysis_sfc_a_250_ref_dif_0.pdf}
            \caption{Control action.}
        \end{subfigure}
        \vskip0.1cm
        \begin{subfigure}[b]{0.475\textwidth}   
            \centering 
            \includegraphics[scale=0.425]{files/sens_analysis/Ref!0/analysis_sfc_a_250_ref_dif_0.pdf}
            \caption{Output.}
        \end{subfigure}
        \caption{Discrete state feedback controller ($r(t)\neq0$) controller for $R_a=250k\Omega$.}
        \label{fig:sens_ra_250_state_noess}
	\end{figure}
\section{Results analysis}\label{sec:resultAn}
\subsection{Model validation}
As the previous section shows, every graph simulated in Simulink presents the same shape as the presented in the original articles, so the model used in this article is valid. Note that there are small differences with the maximum and minimum values that the voltages have and some trajectories that do not fit with the compared graphs. These differences occur due to several reasons that do not diminish the validation of the model. It is considered important to explain the reasons this differences may take place.

In first place, chaotic systems are highly sensitive to small changes in the parameters and initial conditions. The problem is when the system is simulated for every iteration it needs information of every state prior; and, if the system simulated is chaotic every minor approximation that the software makes for a certain iteration influence the posterior iterations value more heavily than other systems. Therefore, the chaotic nature of the attractor makes it difficult to replicate exact graphs from other work.

In second place, other articles do not specify what algorithm they used to solve their equation and, as it is seen in section \ref{subsec:methods}, the algorithm used to solve the differential equation affects the solution (specially if the system is chaotic as the equations used in this paper). And even if the algorithm is specified in the article, like in \cite{rossler1976equation}, the author do not specify the step that he used in the algorithm that also changes the simulation as seen in SECTION EULERDIFERENTEPASO.

Therefore, the problems explained are not related to the implementation of the model worked instead problems inherent to simulation of chaotic systems. Consequently, the model is validated. 

%%%%%%%%%%%%%%%%%%%%%%%%%%%%%%%%%%%%%%%%%%%%%%%%%%%%%%%%%%%%%%%%%%%%%%%%

\subsection{Input Variation}
\subsubsection{Sines Inputs}
As it was presented in section \ref{subsubsec:sines}, both sine inputs make the system fall into a periodic orbit; this is due to the system no having enough time to create a chaotic behavior. It is constantly changing values from voltages that are known to cause chaotic behavior and some that do not. As the sine wave has an offset of $500V$, the average value is exactly $500V$; hence, as the input is much higher than the original $V_{cc0}$ and then the circuit charges and discharges quite often.

Between the two simulations performed (Figs. \ref{fig:3dSin2f} and \ref{fig:3dSin5f}), the sine input with less frequency ($2\text{rad}*s^{-1}$) could achieve a ``simpler'' orbit, since the system had a bit more time to attempt to follow an orbit alike the reference (Figs. \ref{fig:RosslerO} and \ref{fig:3DRosslerO}; unlike the input with higher frequency, where the system immediately falls into a more complex orbit.

\subsubsection{Step Input}\label{subsubsec:stepAn}
As it is shown in section \ref{subsubsec:step}, the step input provides an effective controller for the system in stationary state. It can be observed that the output signal Fig. \ref{fig:outStep} stabilizes shortly after the input is applied; and as well for the phase portrait \ref{fig:3dStep}, since it converges to a point in space. The stabilization is a result of the circuit charging immediately: the voltage is so high that it will not create dynamic behavior through the nodes; it is constant.


%%%%%%%%%%%%%%%%%%%%%%%%%%%%%%%%%%%%%%%%%%%%%%%%%%%%%%%%%%%%%%%%%%%%%%%%

\subsection{Parameter Variation}
It is important to highlight, that for the variation the parameters the values picked cannot be unusually big or small as resistance have a physical limitation for their values. In this manner, there is a threshold for what the minimum and maximum values can the resistances have without breaking the limitations of the system.

\subsubsection{Parameter \texorpdfstring{$R_a$}{Ra}}
It is seen that as the value for the resistance $a$ increases, the system changes for a more chaotic response to a periodic one as seen in figure \ref{fig:paraAvar}. This result does not come as a surprise, as it is consistent with the known circuits theory. As $R_a$ takes higher values, the current that moves around the circuit is lower due to the high resistance that the system has because of Ohm's Law. Therefore, the differences between the voltages in different times of the nodes are smaller, making the system periodic. Consequently, resistance $a$ acts as a controller as for increasing values of this state variable the system easily starts to change its chaotic nature.

This characteristic is more evident when the system is simulated with the same parameters but, increasing the time of simulation as in figure \ref{fig:ra500}. In this figure, the simulation is made with exactly the same parameters as in \ref{fig:3dparaAvard} but with the time of simulation 5 times the original. And, as it is seen, the graphic is totally identical to the original; then, this resistance makes the system completely oscillatory even for longer times therefore completely controlling it for increasing values.

\begin{figure}[H]
    \centering
    \includegraphics[scale=0.5]{figs/ra3652t500.pdf}
    \caption{3D results for $R_a4$ with final time 500.}
    \label{fig:ra500}
\end{figure}

\subsubsection{Parameter \texorpdfstring{$R_c$}{Rc}}
For decreasing values of $R_c$ it is seen that the values of the output increase. This is consistent with Ohm's law, as with lower resistance in the system the voltage is higher as there's more current in the circuit. In this manner, the first signal for $y$ takes more time to show; however, the output signal, as stated in \cite{canals2014random}, is difficult to predict therefore understanding the nature of why this occur is almost impossible.

On the other hand, when the value of this resistance is increased it occurs a phenomenon similar with $R_a$ but, instead of having an immediate periodic behaviour it has some high peaks at the start and then proceeds to control itself as in \ref{fig:paraCvarUpd}. As stated before, the output signal is difficult to predict therefore is not possible to attribute why this peaks happen. Still, it is consistent that the voltage of the output becomes periodic and with small peaks for the same reason that with $R_a$.

Taking this into account, this resistance is harder to predict how it affects the system at the start but, when time passes it takes the expected behaviour. This happens because these parameter affects the non-linear equation, therefore changes done to its value does not make linear changes. On the other hand, $R_a$ changes a linear equation making it quite easy to predict the changes that this value makes to the system.

%%%%%%%%%%%%%%%%%%%%%%%%%%%%%%%%%%%%%%%%%%%%%%%%%%%%%%%%%%%%%%%%%%%%%%%%

\subsection{Limit for Parameter Variation}

\subsubsection{Parameter \texorpdfstring{$R_a$}{Ra}}
As it is seen in \ref{fig:outParaAdown} the output $y$ have a increasing value, tending to infinity. This makes sense, as the resistance $a$ is really low, therefore the same argument is applied as the one in the parameter variation. A analogous argument can be applied for the upper bound. 

The problem with this found interval, is that it is known that this interval presents a range of values for which the system converges but, it is not sure that every value inside this range has a chaotic behavior. To found this interval it would be needed to make a partition of this interval with graphs with similar behaviour and, observe how chaotic is it (calculating the Lyapunov's Exponent) and categorizing it. But, the interval $[259.098131084k\Omega,10000k\Omega]$ is a superset for the set of the values of $R_a$ that makes the system keep its chaotic behaviour.

\subsubsection{Parameter \texorpdfstring{$R_b$}{Rc}}
With this parameter, it can be argued similarly as the one before.

%%%%%%%%%%%%%%%%%%%%%%%%%%%%%%%%%%%%%%%%%%%%%%%%%%%%%%%%%%%%%%%%%%%%%%%%
\subsection{Euler's and Runge-Kutta's Methods}
Section \ref{subsec:methods} shows the simulation results for both methods in both scenarios; Fig. \ref{fig:MethodMethod}. a) shows that, for the Euler method, the numerical solutions for the state equation are equivalent: both scenarios showed identical results since the output signals overlap.

For the Runge-Kutta's method, the results obtained are quite different. As Fig. \ref{fig:MethodMethod} b) shows, both methods match on the peaks localization, but differ on their magnitude. Roughly, for $30s<t<70s$, the difference can be easily perceived.

As it has been already explained, the main feature of chaotic systems is their sensitivity to small changes; the Runge-Kutta's algorithm has to do significantly more calculations, which implies that the propagation of error is much more notorious. As Simulink's implementation of this method is not known, it is tough to determine exactly what causes the difference with further implications; moreover, Simulink's method must have better handling of data and operations, since it must take advantage of the software features, and the algorithm implemented lacks efficiency.


\section{Conclusions}\label{sec:conc}
In this work, a linear control to the Rössler circuit proposed by \cite{canals2014random} was successfully designed with the procedures presented in section \ref{sec:meth}, and then tested under different conditions. Although, some approaches (Ziegler-Nichols and Chien-Hrones-Reswick) did not provide good approximations for the discrete PID controller, the sensitivity and analytic methods showed proper behavior for certain intervals of the inputs and the $R_a$ parameter, usually with an upper bound of $u_{ub} = 1.5$ and $R_a = 3000$, achieving absolute stability and eliminating steady-state error for a step and ramp input.

\nocite{*}
\bibliography{ref}
\bibliographystyle{IEEEtran}
\end{document}
