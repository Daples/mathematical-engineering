\documentclass[a4paper,12pt]{article}
\usepackage[hmargin=2cm,top=4cm,headheight=65pt,footskip=45pt]{geometry}
\usepackage[utf8]{inputenc}
\usepackage{graphicx}
\usepackage[hidelinks]{hyperref}
\usepackage{array}
\usepackage{lastpage}
\usepackage{lipsum}
\usepackage{fancyvrb}
\usepackage{color}
\usepackage{fancyhdr}
\usepackage{amsmath}
\usepackage{enumitem}
\usepackage{titlesec}
\usepackage{floatrow}
\usepackage{float}
\usepackage{subcaption}
\usepackage{caption}
\newfloatcommand{capbtabbox}{table}[][\FBwidth]

\definecolor{customGray}{RGB}{128,128,128}
\definecolor{Eblue}{rgb}{0.0, 0.18, 0.39}
%==============Header & Footnote==============

\pagestyle{fancy}
\renewcommand{\headrulewidth}{0pt}
\fancyhead[C,CO,L,LO,R,RO]{}
\fancyhead[C]{%
          \begin{tabular}{|m{3.0cm}|m{10.0cm}|m{2.5cm}|}
          \hline
          \centering\vspace{1.75mm}\includegraphics[scale=0.275]{logo.pdf} &
          \centering
          {\footnotesize {\sf UNIVERSIDAD EAFIT\\ SCHOOL OF ENGINEERING\\
          \vspace{-1mm}DEPARTMENT OF SYSTEMS AND INFORMATICS}} &
          \centering
          \footnotesize{Page \thepage\ de \pageref{LastPage}\\
          ST245\\
          \vspace{-0.75mm}Data Structures
          }\tabularnewline
          \hline
          \end{tabular}
}
\fancyfoot[C,CO,L,LO,R,RO]{}
\fancyfoot[C]{
          \begin{centering}
            \textcolor{customGray}{{\footnotesize {\sf Professor Mauricio Toro Bermúdez\\
            Phone: $(+57) (4) 261 95 00$ Ext. $9473$. Office: $19 - 627$\\
            \vspace{-1mm}E-mail: mtorobe@eafit.edu.co}}}
        \end{centering}
}

%=============CustomEnumItem===========

\setlist[enumerate]{label=\color{Eblue}\textbf{\roman*.}}

%=============CustomSecSubSec==========

\titleformat{\section}[hang]
{\normalsize\bfseries\itshape\color{black}}{\bfseries\itshape\color{Eblue}\thesection)}{2.5mm}{}

\titleformat{\subsection}[hang]
{\normalsize\bfseries\itshape\color{black}}{\bfseries\color{Eblue}\thesection.\alph{subsection}.}{2.5mm}{}

%==============Title==============

\title{\color{Eblue}\textbf{Laboratory practice No. 3: LinkedList and ArrayList}}
\author{
  \textbf{Juan S. Cárdenas Rodríguez}\\
  Universidad EAFIT\\
  Medellín, Colombia\\
  jscardenar@eafit.edu.co
\and
  \textbf{David Plazas Escudero}\\
  Universidad EAFIT\\
  Medellín, Colombia\\
  dplazas@eafit.edu.co
}

%=============Document=============
\begin{document}
  \maketitle
  \thispagestyle{fancy}

  \section{CODE FOR DELIVERING ON GITHUB}
  The source code can be found in \texttt{Code.py} inside the \texttt{codigo} folder.

  \section{ONLINE EXERCISES}
  The source code can be found in \texttt{Code.py} inside the \texttt{codigo} folder.

  \section{SIMULATION OF PROYECT PRESENTATION QUESTIONS}
    \subsection{Complexity of algorithms using ArrayList and LinkedList}
      \begin{table}[H]
      \centering
      \caption{Complexity for algorithms for ArrayList and LinkedList}
        \begin{tabular}{ccc}
        \hline
          \textbf{}             & \textbf{ArrayList} & \textbf{LinkedList} \\ \hline
          \textbf{Exercise 1.1} & $O(n)$             & $O(n)$              \\
          \textbf{Exercise 1.2} & $O(n)$             & $O(n)$              \\
          \textbf{Exercise 1.3} & $O(n^2)$           & $O(n^3)$            \\
          \textbf{Exercise 1.4} & $O(n^2m)$          & $O(nm)$             \\ \hline
        \end{tabular}
      \end{table}
    \subsection{How does exercise 2.1 work?}
      Exercise 2.1 is simple. First, every time that it finds the character '[', it assigns 0
      to the variable \texttt{index}; representing that it will insert in the beginnig from that point.
      On the other hand, if it finds a ']', \texttt{index} is assigned with the end of the array.

      In any other case, when it finds a character different from '[' and ']', it just adds this characters
      at the \texttt{index} position; therefore, it adds 1 to the \texttt{index} to keep it at the end or at the start.

    \subsection{What's the complexity of exercise 2.1?}
      \begin{Verbatim}
        def manage_string(string):              # c0
            ll = LinkedList()                   # c1
            array = string.split("\n")          # c2
            resp = ""                           # c3
            for line in array:                  # c4*m
                ll.clear()                      # c5*m
                index = 0                       # c6*m
                for char in line:               # c7*m*n
                    if char == "[":             # c8*m*n
                        index = 0               # c9*m*n
                    elif char == "]":           # c10*m*n
                        index = ll.size()       # c11*m*n
                    else:                       # c12*m*n
                        ll.insert(char, index)  # c13*m*n
                        index += 1              # c14*m*n
                for item in ll:                 # c15*n
                    resp += item                # c16*n
                resp += "\n"                    # c17
            return resp                         # c18
      \end{Verbatim}
      Therefore, \texttt{manage\_string} is $O(c_{k1}+c_{k2}m+c_{k3}n+c_{k4}mn)$.
      When the product and sum properties are applied, \texttt{manage\_string} is $O(mn)$,
      where n is the length of the string and m is the number of lines.



  \section{TEST SIMULATION}
    \begin{enumerate}
      \item c)
      \item c)
      \item \begin{itemize}
        \item q.size()
        \item <=
        \item q.remove()
        \item q.remove()
    \end{itemize}
    \end{enumerate}


\end{document}
