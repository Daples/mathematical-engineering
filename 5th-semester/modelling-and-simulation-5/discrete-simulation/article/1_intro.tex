\section{INTRODUCTION}
\label{sec:intro}
Health care is one of the few systems that involve the great majority of the population, therefore there is an increasing demand on all the provided services by health institutions. This is the case of the system analyzed in this work: medication retrieval. These kinds of systems have complex interactions and often evolve into long queues with large waiting times, which creates dissatisfaction among the patients. Thus, health institutions have to face this issue and strive for solutions.

Among the different approaches to treat this problem, Discrete-Event Simulation (DES) emerges as a possible solution. Different authors, such as \cite{thorwarth2009application} and \cite{jun1999application}, provide a review of DES applied to health care. For the specific case of pharmacies, \cite{reynolds2011using} developed a DES model to analyze the automatization of some processes. More references can be found in the references within the mentioned work. Some more references in applications of queue theory in health care (and some in pharmacy) can be found in the excellent review provided by \cite{lakshmi2013application}.


Following the ideas of the above-presented authors, we propose a DES model for the medication retrieval process in the Colombian context, focusing on queue logistics. The main justification for this methodology is that it provides a tool for experimentation without having to use the real system, it takes less time and does not bring risks \cite{reynolds2011using}.


