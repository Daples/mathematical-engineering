\section{BACKGROUND}
In \cite{nejad2008agent}, the authors propose an agent-based simulation (ABS) in order to assist patients to understand the effects of uncontrolled sugar and insulin levels. The paper transforms input variables such as food, exercise, medication, age, ethnicity, and gender into outputs such as blood glucose and blood pressure. This work confirms that there is a direct relation between these factors and that ABS is an adequate tool for modeling diabetes.

Furthermore, \cite{martinez2012definition} proposes an ABS model that represents a fragment of a pancreas of a mouse. Two types of cells are modeled as agents in a three-dimensional space. The two described papers suggest that diabetes is suitable to be analyzed from different approaches, from a really wide perspective, such as general attributes of an individual, to really specific components of the insulin dynamics.

Moreover, \cite{montagna2017agent} introduces an ABS to study and monitor the state of the patient's health and provide suggestions for self-management. 

Finally, \cite{dubovi2019instructional} presents a study of how an AB model can improve students learning, specifically, in the context of diabetes. The study concludes that the implemented model, in NetLogo, improves the learning quality, mostly due to real-time simulation and the graphical interface. This supports the implementation in NetLogo.