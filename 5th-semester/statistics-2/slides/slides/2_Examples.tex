\section{EJEMPLOS}
\begin{frame}{EJEMPLOS}

\textbf{Clasificación entre hombres y mujeres midiendo algunos rasgos psicológicos.} \\\vspace{0.5cm}
Se tienen 32 observaciones de cuatro diferentes factores psicológicos medidos para hombres y mujeres. Diga si la siguiente observación pertenece al grupo masculino o femenino:
    \begin{equation}
        \mathbf{y}=\begin{bmatrix}
            11\\
            17\\
            15\\
            23
        \end{bmatrix}
    \end{equation}

Implementación en R.
\end{frame}

\begin{frame}{EJEMPLOS}
\textbf{Medidas de la cabeza de jugadores de fútbol americano.} \\\vspace{0.5cm}
Se tienen 30 medidas de jugadores de fútbol americano de bachillerato, de universidad y de personas que no juegan fútbol americano sobre 6 aspectos de la cabeza tales como ancho, longitud y más. Construya una función que clasifique a que grupo pertenece una observación nueva y halle su matriz de clasificación.

Implementación en R.
\end{frame}