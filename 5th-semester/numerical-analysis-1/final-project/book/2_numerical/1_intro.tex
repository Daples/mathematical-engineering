\chapter{Numerical Methods}
When treating complex systems, complicated differential equations appear; some of them cannot be solved using analytic methods and others' solutions cannot be expressed in terms of elementary functions, leaving uncertainty on how the solution behaves. In order to give a glimpse of the system's response, numerical methods have been developed. Even though there are plenty of numerical methods to solve ordinary differential equations, in this paper we will be using a few of the most representatives.

On the other hand, when treating fractional differential equations, the classical numerical methods fail. As it was previously stated, fractional derivatives are non-local operators; this implies that the solution of a fractional differential equation at a point not only depends on a few previous points but \textbf{all} previous points. This is often called the ``memory'' of fractional derivatives.

This chapter begins presenting some of the algorithms to solve ordinary differential equations are explained and finally we give an introduction to two general numerical schemes to solve fractional ordinary differential equations. All methods presented here have a short introduction to the deduction of the method or visualization, and some example to show the algorithm's output. The last part of this chapter is focused on giving a short comparison between the algorithms here presented.