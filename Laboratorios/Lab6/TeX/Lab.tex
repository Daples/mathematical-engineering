\documentclass[a4paper,12pt]{article}
\usepackage[hmargin=2cm,top=4cm,headheight=65pt,footskip=45pt]{geometry}
\usepackage[utf8]{inputenc}
\usepackage{graphicx}
\usepackage[hidelinks]{hyperref}
\usepackage{array}
\usepackage{lastpage}
\usepackage{lipsum}
\usepackage{fancyvrb}
\usepackage{color}
\usepackage{fancyhdr}
\usepackage{amsmath}
\usepackage{enumitem}
\usepackage{titlesec}
\usepackage{floatrow}
\usepackage{float}
\usepackage{subcaption}
\usepackage{caption}
\usepackage{multicol}
\newfloatcommand{capbtabbox}{table}[][\FBwidth]

\definecolor{customGray}{RGB}{128,128,128}
\definecolor{Eblue}{rgb}{0.0, 0.18, 0.39}
%==============Header & Footnote==============

\pagestyle{fancy}
\renewcommand{\headrulewidth}{0pt}
\fancyhead[C,CO,L,LO,R,RO]{}
\fancyhead[C]{%
          \begin{tabular}{|m{3.0cm}|m{10.0cm}|m{2.5cm}|}
          \hline
          \centering\vspace{1.75mm}\includegraphics[scale=0.275]{logo.pdf} &
          \centering
          {\footnotesize {\sf UNIVERSIDAD EAFIT\\ SCHOOL OF ENGINEERING\\
          \vspace{-1mm}DEPARTMENT OF SYSTEMS AND INFORMATICS}} &
          \centering
          \footnotesize{Page \thepage\ de \pageref{LastPage}\\
          ST245\\
          \vspace{-0.75mm}Data Structures
          }\tabularnewline
          \hline
          \end{tabular}
}
\fancyfoot[C,CO,L,LO,R,RO]{}
\fancyfoot[C]{
          \begin{centering}
            \textcolor{customGray}{{\footnotesize {\sf Professor Mauricio Toro Bermúdez\\
            Phone: $(+57) (4) 261 95 00$ Ext. $9473$. Office: $19 - 627$\\
            \vspace{-1mm}E-mail: mtorobe@eafit.edu.co}}}
        \end{centering}
}

%=============CustomEnumItem===========

\setlist[enumerate]{label=\color{Eblue}\textbf{\roman*.}}

%=============CustomSecSubSec==========

\titleformat{\section}[hang]
{\normalsize\bfseries\itshape\color{black}}{\bfseries\itshape\color{Eblue}\thesection)}{2.5mm}{}

\titleformat{\subsection}[hang]
{\normalsize\bfseries\itshape\color{black}}{\bfseries\color{Eblue}\thesection.\alph{subsection}.}{2.5mm}{}

%==============Title==============

\title{\color{Eblue}\textbf{Laboratory practice No. 5: Binary Trees}}
\author{
  \textbf{Juan S. Cárdenas Rodríguez}\\
  Universidad EAFIT\\
  Medellín, Colombia\\
  jscardenar@eafit.edu.co
\and
  \textbf{David Plazas Escudero}\\
  Universidad EAFIT\\
  Medellín, Colombia\\
  dplazas@eafit.edu.co
}

%=============Document=============
\begin{document}
  \maketitle
  \thispagestyle{fancy}

  \section{CODE FOR DELIVERING ON GITHUB}
  The source code can be found in \texttt{ParentsTree.java} inside the \texttt{codigo} folder;
  everything is tested in the main method.

  \section{ONLINE EXERCISES}
  The source code can be found in \texttt{Code.java} inside the \texttt{codigo} folder.

  \section{SIMULATION OF PROYECT PRESENTATION QUESTIONS}
    \subsection{Parents Tree??????????}
      You can find binary tree in \texttt{ParentsTree.java}, for David's family.
      Here's the original tree, you can find it implemented in the \texttt{.java} file.
      \begin{Verbatim}
                        David
                      /      \
                     /        \
                    /          \
                   /            \
            Claudia              Gustavo
          /        \              /     \
        /           \            /       \
      Ilva       Efrain    Astrid         Jose
                          /     \
                   Leonilde     Epaminondas
      \end{Verbatim}

    \subsection{Is it possible to implement a better binary tree for parentsTree?}

    \subsection{How does exercise 2.1 work?}
    This exercise is really straight-forward because, when we write a tree in
    pre-order, we immediately know that the first integer is the root, the next
    integers are the left ones and, then the right ones. So, first we insert the
    first integer which is the root; then, as the tree is a binary search tree (BST)
    we know that if we insert them in another BST, it will insert them in the right
    order if and only if we insert them in the order they are given. So, just
    inserting them as the user passes them solves this problem.

    \subsection{What's the complexity of exercise 2.1?}
      \begin{Verbatim}
        public class Code {
            public static void main(String[] args) {
                Scanner sc = new Scanner(System.in); //c1
                System.out.println("Write quit to end the program"); //c2
                String input = sc.next(); //c3
                BinaryTree bt = new BinaryTree(); //c4
                while(!input.equals("quit")) { // c5*n
                    int node = Integer.parseInt(input); // c6*n
                    bt.insert(node); // c7*n*logn
                    input = sc.next(); //c8*n
                }
                bt.posOrder(); // O(n)
            }
        }
      \end{Verbatim}
      Therefore, exercise 2.1 is $O(k_0+k_1n+kn\log{n})$. When the sum and product
      properties of the Big-O notation are applied, exercise 2.1 is $O(n\log{n})$.

  \section{TEST SIMULATION}
    \begin{multicols}{2}
    \begin{enumerate}
      \item \begin{enumerate}
        \item \texttt{altura(raiz.izq)+1}
        \item \texttt{altura(raiz.der)+1}
    \end{enumerate}
      \item c) 3
      \item \begin{enumerate}
          \item $1 == 2$
          \item 0
          \item \texttt{a.izq, suma-a.dato}
          \item \texttt{a.der, suma-a.dato}
    \end{enumerate}
      \item \begin{enumerate}
        \item c) $T(n)=2T(n/2)+C$
        \item a) $O(n)$
        \item d)
        \item a)
    \end{enumerate}
      \item \begin{enumerate}
          \item $p.data == toInsert$
          \item $p.data > toInsert$
    \end{enumerate}
    \end{enumerate}
  \end{multicols}

\end{document}
